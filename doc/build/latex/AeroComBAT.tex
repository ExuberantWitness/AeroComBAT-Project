% Generated by Sphinx.
\def\sphinxdocclass{report}
\documentclass[letterpaper,10pt,english]{sphinxmanual}
\usepackage[utf8]{inputenc}
\DeclareUnicodeCharacter{00A0}{\nobreakspace}
\usepackage{cmap}
\usepackage[T1]{fontenc}
\usepackage{babel}
\usepackage{times}
\usepackage[Bjarne]{fncychap}
\usepackage{longtable}
\usepackage{sphinx}
\usepackage{multirow}

\addto\captionsenglish{\renewcommand{\figurename}{Fig. }}
\addto\captionsenglish{\renewcommand{\tablename}{Table }}
\floatname{literal-block}{Listing }



\title{AeroComBAT Documentation}
\date{January 22, 2016}
\release{0.2.1}
\author{Ben Names}
\newcommand{\sphinxlogo}{}
\renewcommand{\releasename}{Release}
\makeindex

\makeatletter
\def\PYG@reset{\let\PYG@it=\relax \let\PYG@bf=\relax%
    \let\PYG@ul=\relax \let\PYG@tc=\relax%
    \let\PYG@bc=\relax \let\PYG@ff=\relax}
\def\PYG@tok#1{\csname PYG@tok@#1\endcsname}
\def\PYG@toks#1+{\ifx\relax#1\empty\else%
    \PYG@tok{#1}\expandafter\PYG@toks\fi}
\def\PYG@do#1{\PYG@bc{\PYG@tc{\PYG@ul{%
    \PYG@it{\PYG@bf{\PYG@ff{#1}}}}}}}
\def\PYG#1#2{\PYG@reset\PYG@toks#1+\relax+\PYG@do{#2}}

\expandafter\def\csname PYG@tok@gd\endcsname{\def\PYG@tc##1{\textcolor[rgb]{0.63,0.00,0.00}{##1}}}
\expandafter\def\csname PYG@tok@gu\endcsname{\let\PYG@bf=\textbf\def\PYG@tc##1{\textcolor[rgb]{0.50,0.00,0.50}{##1}}}
\expandafter\def\csname PYG@tok@gt\endcsname{\def\PYG@tc##1{\textcolor[rgb]{0.00,0.27,0.87}{##1}}}
\expandafter\def\csname PYG@tok@gs\endcsname{\let\PYG@bf=\textbf}
\expandafter\def\csname PYG@tok@gr\endcsname{\def\PYG@tc##1{\textcolor[rgb]{1.00,0.00,0.00}{##1}}}
\expandafter\def\csname PYG@tok@cm\endcsname{\let\PYG@it=\textit\def\PYG@tc##1{\textcolor[rgb]{0.25,0.50,0.56}{##1}}}
\expandafter\def\csname PYG@tok@vg\endcsname{\def\PYG@tc##1{\textcolor[rgb]{0.73,0.38,0.84}{##1}}}
\expandafter\def\csname PYG@tok@m\endcsname{\def\PYG@tc##1{\textcolor[rgb]{0.13,0.50,0.31}{##1}}}
\expandafter\def\csname PYG@tok@mh\endcsname{\def\PYG@tc##1{\textcolor[rgb]{0.13,0.50,0.31}{##1}}}
\expandafter\def\csname PYG@tok@cs\endcsname{\def\PYG@tc##1{\textcolor[rgb]{0.25,0.50,0.56}{##1}}\def\PYG@bc##1{\setlength{\fboxsep}{0pt}\colorbox[rgb]{1.00,0.94,0.94}{\strut ##1}}}
\expandafter\def\csname PYG@tok@ge\endcsname{\let\PYG@it=\textit}
\expandafter\def\csname PYG@tok@vc\endcsname{\def\PYG@tc##1{\textcolor[rgb]{0.73,0.38,0.84}{##1}}}
\expandafter\def\csname PYG@tok@il\endcsname{\def\PYG@tc##1{\textcolor[rgb]{0.13,0.50,0.31}{##1}}}
\expandafter\def\csname PYG@tok@go\endcsname{\def\PYG@tc##1{\textcolor[rgb]{0.20,0.20,0.20}{##1}}}
\expandafter\def\csname PYG@tok@cp\endcsname{\def\PYG@tc##1{\textcolor[rgb]{0.00,0.44,0.13}{##1}}}
\expandafter\def\csname PYG@tok@gi\endcsname{\def\PYG@tc##1{\textcolor[rgb]{0.00,0.63,0.00}{##1}}}
\expandafter\def\csname PYG@tok@gh\endcsname{\let\PYG@bf=\textbf\def\PYG@tc##1{\textcolor[rgb]{0.00,0.00,0.50}{##1}}}
\expandafter\def\csname PYG@tok@ni\endcsname{\let\PYG@bf=\textbf\def\PYG@tc##1{\textcolor[rgb]{0.84,0.33,0.22}{##1}}}
\expandafter\def\csname PYG@tok@nl\endcsname{\let\PYG@bf=\textbf\def\PYG@tc##1{\textcolor[rgb]{0.00,0.13,0.44}{##1}}}
\expandafter\def\csname PYG@tok@nn\endcsname{\let\PYG@bf=\textbf\def\PYG@tc##1{\textcolor[rgb]{0.05,0.52,0.71}{##1}}}
\expandafter\def\csname PYG@tok@no\endcsname{\def\PYG@tc##1{\textcolor[rgb]{0.38,0.68,0.84}{##1}}}
\expandafter\def\csname PYG@tok@na\endcsname{\def\PYG@tc##1{\textcolor[rgb]{0.25,0.44,0.63}{##1}}}
\expandafter\def\csname PYG@tok@nb\endcsname{\def\PYG@tc##1{\textcolor[rgb]{0.00,0.44,0.13}{##1}}}
\expandafter\def\csname PYG@tok@nc\endcsname{\let\PYG@bf=\textbf\def\PYG@tc##1{\textcolor[rgb]{0.05,0.52,0.71}{##1}}}
\expandafter\def\csname PYG@tok@nd\endcsname{\let\PYG@bf=\textbf\def\PYG@tc##1{\textcolor[rgb]{0.33,0.33,0.33}{##1}}}
\expandafter\def\csname PYG@tok@ne\endcsname{\def\PYG@tc##1{\textcolor[rgb]{0.00,0.44,0.13}{##1}}}
\expandafter\def\csname PYG@tok@nf\endcsname{\def\PYG@tc##1{\textcolor[rgb]{0.02,0.16,0.49}{##1}}}
\expandafter\def\csname PYG@tok@si\endcsname{\let\PYG@it=\textit\def\PYG@tc##1{\textcolor[rgb]{0.44,0.63,0.82}{##1}}}
\expandafter\def\csname PYG@tok@s2\endcsname{\def\PYG@tc##1{\textcolor[rgb]{0.25,0.44,0.63}{##1}}}
\expandafter\def\csname PYG@tok@vi\endcsname{\def\PYG@tc##1{\textcolor[rgb]{0.73,0.38,0.84}{##1}}}
\expandafter\def\csname PYG@tok@nt\endcsname{\let\PYG@bf=\textbf\def\PYG@tc##1{\textcolor[rgb]{0.02,0.16,0.45}{##1}}}
\expandafter\def\csname PYG@tok@nv\endcsname{\def\PYG@tc##1{\textcolor[rgb]{0.73,0.38,0.84}{##1}}}
\expandafter\def\csname PYG@tok@s1\endcsname{\def\PYG@tc##1{\textcolor[rgb]{0.25,0.44,0.63}{##1}}}
\expandafter\def\csname PYG@tok@gp\endcsname{\let\PYG@bf=\textbf\def\PYG@tc##1{\textcolor[rgb]{0.78,0.36,0.04}{##1}}}
\expandafter\def\csname PYG@tok@sh\endcsname{\def\PYG@tc##1{\textcolor[rgb]{0.25,0.44,0.63}{##1}}}
\expandafter\def\csname PYG@tok@ow\endcsname{\let\PYG@bf=\textbf\def\PYG@tc##1{\textcolor[rgb]{0.00,0.44,0.13}{##1}}}
\expandafter\def\csname PYG@tok@sx\endcsname{\def\PYG@tc##1{\textcolor[rgb]{0.78,0.36,0.04}{##1}}}
\expandafter\def\csname PYG@tok@bp\endcsname{\def\PYG@tc##1{\textcolor[rgb]{0.00,0.44,0.13}{##1}}}
\expandafter\def\csname PYG@tok@c1\endcsname{\let\PYG@it=\textit\def\PYG@tc##1{\textcolor[rgb]{0.25,0.50,0.56}{##1}}}
\expandafter\def\csname PYG@tok@kc\endcsname{\let\PYG@bf=\textbf\def\PYG@tc##1{\textcolor[rgb]{0.00,0.44,0.13}{##1}}}
\expandafter\def\csname PYG@tok@c\endcsname{\let\PYG@it=\textit\def\PYG@tc##1{\textcolor[rgb]{0.25,0.50,0.56}{##1}}}
\expandafter\def\csname PYG@tok@mf\endcsname{\def\PYG@tc##1{\textcolor[rgb]{0.13,0.50,0.31}{##1}}}
\expandafter\def\csname PYG@tok@err\endcsname{\def\PYG@bc##1{\setlength{\fboxsep}{0pt}\fcolorbox[rgb]{1.00,0.00,0.00}{1,1,1}{\strut ##1}}}
\expandafter\def\csname PYG@tok@mb\endcsname{\def\PYG@tc##1{\textcolor[rgb]{0.13,0.50,0.31}{##1}}}
\expandafter\def\csname PYG@tok@ss\endcsname{\def\PYG@tc##1{\textcolor[rgb]{0.32,0.47,0.09}{##1}}}
\expandafter\def\csname PYG@tok@sr\endcsname{\def\PYG@tc##1{\textcolor[rgb]{0.14,0.33,0.53}{##1}}}
\expandafter\def\csname PYG@tok@mo\endcsname{\def\PYG@tc##1{\textcolor[rgb]{0.13,0.50,0.31}{##1}}}
\expandafter\def\csname PYG@tok@kd\endcsname{\let\PYG@bf=\textbf\def\PYG@tc##1{\textcolor[rgb]{0.00,0.44,0.13}{##1}}}
\expandafter\def\csname PYG@tok@mi\endcsname{\def\PYG@tc##1{\textcolor[rgb]{0.13,0.50,0.31}{##1}}}
\expandafter\def\csname PYG@tok@kn\endcsname{\let\PYG@bf=\textbf\def\PYG@tc##1{\textcolor[rgb]{0.00,0.44,0.13}{##1}}}
\expandafter\def\csname PYG@tok@o\endcsname{\def\PYG@tc##1{\textcolor[rgb]{0.40,0.40,0.40}{##1}}}
\expandafter\def\csname PYG@tok@kr\endcsname{\let\PYG@bf=\textbf\def\PYG@tc##1{\textcolor[rgb]{0.00,0.44,0.13}{##1}}}
\expandafter\def\csname PYG@tok@s\endcsname{\def\PYG@tc##1{\textcolor[rgb]{0.25,0.44,0.63}{##1}}}
\expandafter\def\csname PYG@tok@kp\endcsname{\def\PYG@tc##1{\textcolor[rgb]{0.00,0.44,0.13}{##1}}}
\expandafter\def\csname PYG@tok@w\endcsname{\def\PYG@tc##1{\textcolor[rgb]{0.73,0.73,0.73}{##1}}}
\expandafter\def\csname PYG@tok@kt\endcsname{\def\PYG@tc##1{\textcolor[rgb]{0.56,0.13,0.00}{##1}}}
\expandafter\def\csname PYG@tok@sc\endcsname{\def\PYG@tc##1{\textcolor[rgb]{0.25,0.44,0.63}{##1}}}
\expandafter\def\csname PYG@tok@sb\endcsname{\def\PYG@tc##1{\textcolor[rgb]{0.25,0.44,0.63}{##1}}}
\expandafter\def\csname PYG@tok@k\endcsname{\let\PYG@bf=\textbf\def\PYG@tc##1{\textcolor[rgb]{0.00,0.44,0.13}{##1}}}
\expandafter\def\csname PYG@tok@se\endcsname{\let\PYG@bf=\textbf\def\PYG@tc##1{\textcolor[rgb]{0.25,0.44,0.63}{##1}}}
\expandafter\def\csname PYG@tok@sd\endcsname{\let\PYG@it=\textit\def\PYG@tc##1{\textcolor[rgb]{0.25,0.44,0.63}{##1}}}

\def\PYGZbs{\char`\\}
\def\PYGZus{\char`\_}
\def\PYGZob{\char`\{}
\def\PYGZcb{\char`\}}
\def\PYGZca{\char`\^}
\def\PYGZam{\char`\&}
\def\PYGZlt{\char`\<}
\def\PYGZgt{\char`\>}
\def\PYGZsh{\char`\#}
\def\PYGZpc{\char`\%}
\def\PYGZdl{\char`\$}
\def\PYGZhy{\char`\-}
\def\PYGZsq{\char`\'}
\def\PYGZdq{\char`\"}
\def\PYGZti{\char`\~}
% for compatibility with earlier versions
\def\PYGZat{@}
\def\PYGZlb{[}
\def\PYGZrb{]}
\makeatother

\renewcommand\PYGZsq{\textquotesingle}

\begin{document}

\maketitle
\tableofcontents
\phantomsection\label{index::doc}


Contents:


\chapter{The FEM Interface Module}
\label{FEM:welcome-to-aerocombat-s-documentation}\label{FEM:module-AeroComBAT.FEM}\label{FEM::doc}\label{FEM:the-fem-interface-module}\index{AeroComBAT.FEM (module)}
This module contains a basic environment for conducting finite element analysis.

The primary purpose of this library is to fascilitate the creation of a FEM
within the AeroComBAT package.
\begin{quote}\begin{description}
\item[{SUMARRY OF THE CLASSES}] \leavevmode
\end{description}\end{quote}
\begin{itemize}
\item {} \begin{description}
\item[{\emph{Model}: The Model class has two main purposes. The first is that it is meant}] \leavevmode
to serve as an organizational class. Once an aircraft part has been loaded
into the model by using the addAircraftPart() method, the aircraft part
can be loaded and constrained by the user. Once all parts have been loaded
into the model and all loads and constraints have been applied, the user
can choose to execute the plotRigidModel() method to visualize the model
and make sure it accurately represents their problem. If the model appears
as it should, the user can elect to run a static, buckling, normal mode,
static aeroelastic, or dynamic flutter analysis.

\end{description}

\end{itemize}

\begin{notice}{note}{Note:}
Currently the only avaliable part in the AeroComBAT package are wing
parts, however this is likely to change as parts such as masses, fuselages
and other types of aircraft parts are added.
\end{notice}
\phantomsection\label{FEM:module-AeroComBAT.FEM}\index{AeroComBAT.FEM (module)}

\section{MODEL}
\label{FEM:model}\index{Model (class in AeroComBAT.FEM)}

\begin{fulllineitems}
\phantomsection\label{FEM:AeroComBAT.FEM.Model}\pysigline{\strong{class }\code{AeroComBAT.FEM.}\bfcode{Model}}
Creates a Model which is used to organize and analyze FEM.

The primary used of Model objects are to organize FEM's and analyze them.
The Model object doesn't create any finite elements. Instead, it loads
aircraft parts which contain various types of finite element structural
models as well as aerodynamic models. The type of model will depend on the
type of aircraft part added. Once all of the models are created and added
to the model object, the model object will serve as the analysis primary
interface used to manipulate the generated model.
\begin{quote}\begin{description}
\item[{Attributes}] \leavevmode
\end{description}\end{quote}
\begin{itemize}
\item {} 
\emph{Kg (DOFxDOF np.array{[}float{]})}: This is the global stiffness matrix.

\item {} \begin{description}
\item[{\emph{Kgr ((DOF-CON)x(DOF-CON) np.array{[}float{]})}: This is the global reduced}] \leavevmode
stiffness matrix. In other words, the global stiffness matrix with the
rows and columns corresponding to the constraints (CON) removed.

\end{description}

\item {} 
\emph{Fg (DOFx1 np.array{[}float{]})}: The global force vector.

\item {} \begin{description}
\item[{\emph{Fgr ((DOF-CON)x1 np.array{[}float{]})}: The global reduced force vector. In}] \leavevmode
other words, the global force vector with the rows corresponding to the
constraints (CON) removed.

\end{description}

\item {} 
\emph{Mg (DOFxDOF np.array{[}float{]})}: The global mass matrix.

\item {} \begin{description}
\item[{\emph{Mgr ((DOF-CON)x(DOF-CON) np.array{[}float{]})}: The global reduced mass}] \leavevmode
matrix. In other words, the global mass matrix with the rows and
columns corresponding to the constraints (CON) removed.

\end{description}

\item {} \begin{description}
\item[{\emph{Qg (DOFx1 np.array{[}float{]})}: The global force boundary condition vector.}] \leavevmode
This is where all of the nodal loads are stored before the system is
assembled.

\end{description}

\item {} \begin{description}
\item[{\emph{nids (Array{[}int{]})}: This array contains all of the node IDs used within}] \leavevmode
the model.

\end{description}

\item {} \begin{description}
\item[{\emph{nodeDict (dict{[}NID,node{]})}: This dictionary is a mapping of the node IDs}] \leavevmode
used within the model to the corresponding node objects.

\end{description}

\item {} \begin{description}
\item[{\emph{elems (Array{[}obj{]})}: This array contains all of the element objects used}] \leavevmode
in the model.

\end{description}

\item {} \begin{description}
\item[{\emph{const (dict{[}NID,Array{[}DOF{]})}: This dictionary is a mapping of the node}] \leavevmode
IDs constrained and the corresponding degrees of freedom that are
constrained.

\end{description}

\item {} \begin{description}
\item[{\emph{parts (Array{[}int{]})}: This array contains all of the part IDs}] \leavevmode
corresponding to the parts that have been added to the model.

\end{description}

\end{itemize}
\begin{quote}\begin{description}
\item[{Methods}] \leavevmode
\end{description}\end{quote}
\begin{itemize}
\item {} 
\emph{addElements}: A method to add individual elements to the model.

\item {} \begin{description}
\item[{\emph{addAircraftParts}: A method to add an Aircraft part to the model. This}] \leavevmode
is a much more effective method than addElements as when a part is
added, the model can utilize all of the organizational and post
processing methods built into the part.

\end{description}

\item {} \begin{description}
\item[{\emph{resetPointLoads}: A convenient way to reset all of the nodal loads in}] \leavevmode
the model to zero.

\end{description}

\item {} \begin{description}
\item[{\emph{resetResults}: A convenient way to clear the results in all of the}] \leavevmode
elements from a previous analysis. This method is subject to change as
the way in which results are stored is likely to change.

\end{description}

\item {} \begin{description}
\item[{\emph{applyLoads}: A method to apply nodal loads as well as distributed loads}] \leavevmode
to a range of elements, all of the elements in a part, or all of the
elements in the model.

\end{description}

\item {} 
\emph{applyConstraints}: A method to apply nodal constraints to the model.

\item {} 
\emph{staticAnalysis}: A method which conducts a linear static analysis.

\item {} \begin{description}
\item[{\emph{normalModesAnalysis}: A method which conducts a normal modes analysis on}] \leavevmode
the model.

\end{description}

\item {} 
\emph{plotRigidModel}: A method to plot and visualize the model.

\item {} \begin{description}
\item[{\emph{plotDeformedModel}: A method to plot and visualize the results from an}] \leavevmode
analysis on the model.

\end{description}

\end{itemize}

\begin{notice}{note}{Note:}
When constraining nodes, only 0 displacement and rotation
constraints are currently supported.
\end{notice}
\index{addAircraftParts() (AeroComBAT.FEM.Model method)}

\begin{fulllineitems}
\phantomsection\label{FEM:AeroComBAT.FEM.Model.addAircraftParts}\pysiglinewithargsret{\bfcode{addAircraftParts}}{\emph{parts}}{}
A method to add an array of aircraft parts to the model.

This method is a robust version of addElements. provided an array of
part objects, this method will add the parts to the model. This
includes adding all of the elements and nodes to the model, as well as
a few other pieces of information.
\begin{quote}\begin{description}
\item[{Args}] \leavevmode
\end{description}\end{quote}
\begin{itemize}
\item {} 
\emph{parts (Array{[}obj{]})}: An array of part objects.

\end{itemize}
\begin{quote}\begin{description}
\item[{Returns}] \leavevmode
\end{description}\end{quote}
\begin{itemize}
\item {} 
None

\end{itemize}

\end{fulllineitems}

\index{addElements() (AeroComBAT.FEM.Model method)}

\begin{fulllineitems}
\phantomsection\label{FEM:AeroComBAT.FEM.Model.addElements}\pysiglinewithargsret{\bfcode{addElements}}{\emph{elemarray}}{}
A method to add elements to the model.

Provided an array of elements, this method can add those elements to
the model for analysis. This is a rather rudementary method as the post
processing methods utilized by the parts are not at the users disposal
for the elements added to the model in this way.
\begin{quote}\begin{description}
\item[{Args}] \leavevmode
\end{description}\end{quote}
\begin{itemize}
\item {} \begin{description}
\item[{\emph{elemarray (Array{[}obj{]})}: Adds all of the elements in the array to}] \leavevmode
the model.

\end{description}

\end{itemize}
\begin{quote}\begin{description}
\item[{Returns}] \leavevmode
\end{description}\end{quote}
\begin{itemize}
\item {} 
None

\end{itemize}

\begin{notice}{note}{Note:}
Currently supported elements include: SuperBeam, Tbeam.
\end{notice}

\end{fulllineitems}

\index{applyConstraints() (AeroComBAT.FEM.Model method)}

\begin{fulllineitems}
\phantomsection\label{FEM:AeroComBAT.FEM.Model.applyConstraints}\pysiglinewithargsret{\bfcode{applyConstraints}}{\emph{NID}, \emph{const}}{}
A method for applying nodal constraints to the model.

This method is the primary method for applying nodal constraints to the
model.
\begin{quote}\begin{description}
\item[{Args}] \leavevmode
\end{description}\end{quote}
\begin{itemize}
\item {} 
\emph{NID (int)}: The node ID of the node to be constrained.

\item {} \begin{description}
\item[{\emph{const (str, np.array{[}int{]})}: const can either take the form of a}] \leavevmode
string in order to take advantage of the two most common
constraints being `pin' or `fix'. If a different constraint needs
to be applied, const could also be a numpy array listing the DOF
(integers 1-6) to be constrained.

\end{description}

\end{itemize}
\begin{quote}\begin{description}
\item[{Returns}] \leavevmode
\end{description}\end{quote}
\begin{itemize}
\item {} 
None

\end{itemize}

\end{fulllineitems}

\index{applyLoads() (AeroComBAT.FEM.Model method)}

\begin{fulllineitems}
\phantomsection\label{FEM:AeroComBAT.FEM.Model.applyLoads}\pysiglinewithargsret{\bfcode{applyLoads}}{\emph{**kwargs}}{}
A method to apply nodal and distributed loads to the model.

This method allows the user to apply nodal loads to nodes and
distributed loads to elements within the model.
\begin{quote}\begin{description}
\item[{Args}] \leavevmode
\end{description}\end{quote}
\begin{itemize}
\item {} \begin{description}
\item[{\emph{f (func)}: A function which, provided the provided a length 3 numpy}] \leavevmode
array representing a point in space, calculates the distributed
load value at that point. See an example below:

\end{description}

\item {} \begin{description}
\item[{\emph{F (dict{[}NID,1x6 np.array{[}float{]}{]})}: A dictionary mapping a node ID}] \leavevmode
to the loads to be applied at that node ID.

\end{description}

\item {} \begin{description}
\item[{\emph{allElems (bool)}: A boolean value used to easily load all of the}] \leavevmode
elements which have been added to the model.

\end{description}

\item {} \begin{description}
\item[{\emph{PIDs (Array{[}int{]})}: An array containing part ID's, signifying that}] \leavevmode
all elements used by that part should be loaded.

\end{description}

\item {} \begin{description}
\item[{\emph{eids (Array{[}int{]})}: An array containing all of the element ID's}] \leavevmode
corresponding to all of the elements which should be loaded.

\end{description}

\end{itemize}
\begin{quote}\begin{description}
\item[{Returns}] \leavevmode
\end{description}\end{quote}
\begin{itemize}
\item {} 
None

\end{itemize}

Distributed load example:
\textgreater{}\textgreater{}\textgreater{}def f(x):
\begin{quote}

vx = (1/10)*10*x{[}2{]}**2-7*x{[}2{]}-2.1
vy = 10*x{[}2{]}**2-7*x{[}2{]}
pz = 0
tz = (10*x{[}2{]}**2-7*x{[}2{]})/10+3*x{[}0{]}**2
return np.array({[}vx,vy,pz,tz{]})
\end{quote}

Nodal load example:
\begin{description}
\item[{::}] \leavevmode
F{[}NID{]} = np.array({[}Qx,Qy,P,Mx,My,T{]})

\end{description}

\end{fulllineitems}

\index{normalModesAnalysis() (AeroComBAT.FEM.Model method)}

\begin{fulllineitems}
\phantomsection\label{FEM:AeroComBAT.FEM.Model.normalModesAnalysis}\pysiglinewithargsret{\bfcode{normalModesAnalysis}}{\emph{**kwargs}}{}
Conducts normal mode analysis.

This method conducts normal mode analysis on the model. This will
calculate all of the unknown frequency eigenvalues and eigenvectors for
the model, which can be plotted later.
\begin{quote}\begin{description}
\item[{Args}] \leavevmode
\end{description}\end{quote}
\begin{itemize}
\item {} \begin{description}
\item[{\emph{analysis\_name (str)}: The string name to be associated with this}] \leavevmode
analysis. By default, this is chosen to be `analysis\_untitled'.

\end{description}

\end{itemize}
\begin{quote}\begin{description}
\item[{Returns}] \leavevmode
\end{description}\end{quote}
\begin{itemize}
\item {} 
None

\end{itemize}

\begin{notice}{note}{Note:}
There are internal loads that are calculated and stored
within the model elements, however be aware that these loads are
meaningless and are only retained as a means to display cross
section warping.
\end{notice}

\end{fulllineitems}

\index{plotDeformedModel() (AeroComBAT.FEM.Model method)}

\begin{fulllineitems}
\phantomsection\label{FEM:AeroComBAT.FEM.Model.plotDeformedModel}\pysiglinewithargsret{\bfcode{plotDeformedModel}}{\emph{**kwargs}}{}
Plots the deformed model.

This method plots the deformed model results for a given analysis in
the mayavi environement.
\begin{quote}\begin{description}
\item[{Args}] \leavevmode
\end{description}\end{quote}
\begin{itemize}
\item {} 
\emph{analysis\_name (str)}: The string identifier of the analysis.

\item {} \begin{description}
\item[{\emph{figName (str)}: The name of the figure. This is `Rigid Model' by}] \leavevmode
default.

\end{description}

\item {} \begin{description}
\item[{\emph{clr (1x3 tuple(int))}: The color tuple or RGB values to be used for}] \leavevmode
plotting the reference axis for all beam elements. By default this
color is black.

\end{description}

\item {} \begin{description}
\item[{\emph{numXSects (int)}: The number of cross-sections desired to be plotted}] \leavevmode
for all wing sections. The default is 2.

\end{description}

\item {} \begin{description}
\item[{\emph{contour (str)}: A string keyword to determine what analysis should}] \leavevmode
be plotted.

\end{description}

\item {} \begin{description}
\item[{\emph{contLim (1x2 Array{[}float{]})}: An array containing the lower and upper}] \leavevmode
contour limits.

\end{description}

\item {} \begin{description}
\item[{\emph{warpScale (float)}: The scaling factor used to magnify the cross}] \leavevmode
section warping displacement factor.

\end{description}

\item {} \begin{description}
\item[{\emph{displScale (float)}: The scaling fator used to magnify the beam}] \leavevmode
element displacements and rotations.

\end{description}

\item {} \begin{description}
\item[{\emph{mode (int)}: If the analysis name refers to a modal analysis, mode}] \leavevmode
refers to which mode from that analysis should be plotted.

\end{description}

\end{itemize}
\begin{quote}\begin{description}
\item[{Returns}] \leavevmode
\end{description}\end{quote}
\begin{itemize}
\item {} 
mayavi figure

\end{itemize}

\end{fulllineitems}

\index{plotRigidModel() (AeroComBAT.FEM.Model method)}

\begin{fulllineitems}
\phantomsection\label{FEM:AeroComBAT.FEM.Model.plotRigidModel}\pysiglinewithargsret{\bfcode{plotRigidModel}}{\emph{**kwargs}}{}
Plots the rigid model.

This method plots the rigid model in the mayavi environement.
\begin{quote}\begin{description}
\item[{Args}] \leavevmode
\end{description}\end{quote}
\begin{itemize}
\item {} \begin{description}
\item[{\emph{figName (str)}: The name of the figure. This is `Rigid Model' by}] \leavevmode
default.

\end{description}

\item {} \begin{description}
\item[{\emph{clr (1x3 tuple(int))}: The color tuple or RGB values to be used for}] \leavevmode
plotting the reference axis for all beam elements. By default this
color is black.

\end{description}

\item {} \begin{description}
\item[{\emph{numXSects (int)}: The number of cross-sections desired to be plotted}] \leavevmode
for all wing sections. The default is 2.

\end{description}

\end{itemize}
\begin{quote}\begin{description}
\item[{Returns}] \leavevmode
\end{description}\end{quote}
\begin{itemize}
\item {} 
mayavi figure

\end{itemize}

\end{fulllineitems}

\index{resetPointLoads() (AeroComBAT.FEM.Model method)}

\begin{fulllineitems}
\phantomsection\label{FEM:AeroComBAT.FEM.Model.resetPointLoads}\pysiglinewithargsret{\bfcode{resetPointLoads}}{}{}
A method to reset the point loads applied to the model.

This is a good method to reset the nodal loads applied to a model. This
method will be useful when attempting to apply a series different
analysis.
\begin{quote}\begin{description}
\item[{Args}] \leavevmode
\end{description}\end{quote}
\begin{itemize}
\item {} 
None

\end{itemize}
\begin{quote}\begin{description}
\item[{Returns}] \leavevmode
\end{description}\end{quote}
\begin{itemize}
\item {} 
None

\end{itemize}

\end{fulllineitems}

\index{resetResults() (AeroComBAT.FEM.Model method)}

\begin{fulllineitems}
\phantomsection\label{FEM:AeroComBAT.FEM.Model.resetResults}\pysiglinewithargsret{\bfcode{resetResults}}{}{}
A method to reset the results in a model.

This is a good method to reset the results in the model from a given
analysis. This method will be useful when attempting to apply a series
different analysis.
\begin{quote}\begin{description}
\item[{Args}] \leavevmode
\end{description}\end{quote}
\begin{itemize}
\item {} 
None

\end{itemize}
\begin{quote}\begin{description}
\item[{Returns}] \leavevmode
\end{description}\end{quote}
\begin{itemize}
\item {} 
None

\end{itemize}

\end{fulllineitems}

\index{staticAnalysis() (AeroComBAT.FEM.Model method)}

\begin{fulllineitems}
\phantomsection\label{FEM:AeroComBAT.FEM.Model.staticAnalysis}\pysiglinewithargsret{\bfcode{staticAnalysis}}{\emph{**kwargs}}{}
Linear static analysis.

This method conducts a linear static analysis on the model. This will
calculate all of the unknown displacements in the model, and save not
only dispalcements, but also internal forces and moments in all of the
beam elements.
\begin{quote}\begin{description}
\item[{Args}] \leavevmode
\end{description}\end{quote}
\begin{itemize}
\item {} \begin{description}
\item[{\emph{analysis\_name (str)}: The string name to be associated with this}] \leavevmode
analysis. By default, this is chosen to be `analysis\_untitled'.

\end{description}

\end{itemize}
\begin{quote}\begin{description}
\item[{Returns}] \leavevmode
\end{description}\end{quote}
\begin{itemize}
\item {} 
None

\end{itemize}

\end{fulllineitems}


\end{fulllineitems}



\chapter{The Structures Module}
\label{structures:the-structures-module}\label{structures::doc}\label{structures:module-AeroComBAT.Structures}\index{AeroComBAT.Structures (module)}
This module contains a library of classes devoted to structural analysis.

The primary purpose of this library is to fascilitate the ROM (reduced order
modeling) of structures that can simplified to beams. The real power of this
library comes from it's the XSect class. This class can create and analyze
a cross-section, allowing the user to accurately model a nonhomogeneous
(made of multiple materials) anisotropic (materials that behave anisotropically
such as composites) complex cross-sections.

It should be noted that classes are ordered by model complexity. The further
down the structures.py library, the more complex the objects, often requiring
multiple of their predecessors. For example, the CQUAD4 class requires four
node objects and a material object.
\begin{quote}\begin{description}
\item[{SUMARRY OF THE CLASSES}] \leavevmode
\end{description}\end{quote}
\begin{itemize}
\item {} 
\emph{Node}: Creates a node object with 3D position.

\item {} 
\emph{Material}: Creates a material object, generating the 3D constitutive relations.

\item {} \begin{description}
\item[{\emph{MicroMechanics}: Class to fascilitate the calculation of composite stiffnesses}] \leavevmode
using micro-mechanical models where fibers are long and continuous.

\end{description}

\item {} \begin{description}
\item[{\emph{CQUAD4}: Creates a 2D linear quadrilateral element, mainly used to fascilitate    cross-sectional analysis, this class could be modified in future updates}] \leavevmode
such that they could also be used to create plate or laminate element
objects as well.

\end{description}

\item {} \begin{description}
\item[{\emph{MaterialLib}: Creates a material library object meant to hold many material}] \leavevmode
objects.

\end{description}

\item {} 
\emph{Ply}: Creates ply objects which are used in the building of a laminate object.

\item {} \begin{description}
\item[{\emph{Laminate}: Creates laminate objects which could be used for CLT (classical}] \leavevmode
lamination theory) analysis as well as to be used in building a beam
cross-section.

\end{description}

\item {} \begin{description}
\item[{\emph{XSect}: Creates a cross-section object which can be used in the ROM of a beam}] \leavevmode
with a non-homogeneous anisotropic cross-section. Currently only supports
simple box beam cross-section (i.e., four laminates joined together to form
a box), however outer mold lines can take the shape of airfoil profiles.
See the Airfoil class in AircraftParts.py for more info.

\end{description}

\item {} 
\emph{TBeam}: Creates a single Timoshenko beam object for FEA.

\item {} \begin{description}
\item[{\emph{SuperBeam}: Creates a super beam object. This class is mainly used to automate}] \leavevmode
the creation of many connected TBeam objects to be used late for FEA.

\end{description}

\item {} \begin{description}
\item[{\emph{WingSection}: A class which creates and holds many super beams, each of which}] \leavevmode
could have different cross-sections. It also helps to dimensionalize
plates for simple closed-form composite buckling load aproximations.

\end{description}

\end{itemize}

\begin{notice}{note}{Note:}
Currently the inclusion of thermal strains are not supported for any
structural model.
\end{notice}
\phantomsection\label{structures:module-AeroComBAT.Structures}\index{AeroComBAT.Structures (module)}

\section{NODE}
\label{structures:node}\index{Node (class in AeroComBAT.Structures)}

\begin{fulllineitems}
\phantomsection\label{structures:AeroComBAT.Structures.Node}\pysiglinewithargsret{\strong{class }\code{AeroComBAT.Structures.}\bfcode{Node}}{\emph{nid}, \emph{x}}{}
Creates a node object.

Node objects could be used in any finite element implementation.
\begin{quote}\begin{description}
\item[{Attributes}] \leavevmode
\end{description}\end{quote}
\begin{itemize}
\item {} 
\emph{NID (int)}: The integer identifier given to the object.

\item {} \begin{description}
\item[{\emph{x (Array{[}float{]})}: An array containing the 3 x-y-z coordinates of the}] \leavevmode
node.

\end{description}

\item {} \begin{description}
\item[{\emph{summary (str)}: A string which is a tabulated respresentation and}] \leavevmode
summary of the important attributes of the object.

\end{description}

\end{itemize}
\begin{quote}\begin{description}
\item[{Methods}] \leavevmode
\end{description}\end{quote}
\begin{itemize}
\item {} \begin{description}
\item[{\emph{printSummary}: This method prints out basic information about the node}] \leavevmode
object, such as it's node ID and it's x-y-z coordinates

\end{description}

\end{itemize}
\index{\_\_init\_\_() (AeroComBAT.Structures.Node method)}

\begin{fulllineitems}
\phantomsection\label{structures:AeroComBAT.Structures.Node.__init__}\pysiglinewithargsret{\bfcode{\_\_init\_\_}}{\emph{nid}, \emph{x}}{}
Initializes the node object.
\begin{quote}\begin{description}
\item[{Args}] \leavevmode
\end{description}\end{quote}
\begin{itemize}
\item {} 
\emph{nid (int)}: The desired integer node ID

\item {} 
\emph{x (Array{[}float{]})}: The position of the node in 3D space.

\end{itemize}

\end{fulllineitems}

\index{printSummary() (AeroComBAT.Structures.Node method)}

\begin{fulllineitems}
\phantomsection\label{structures:AeroComBAT.Structures.Node.printSummary}\pysiglinewithargsret{\bfcode{printSummary}}{}{}
Prints basic information about the node.

The printSummary method prints out basic node attributes in an organized
fashion. This includes the node ID and x-y-z global coordinates.
\begin{quote}\begin{description}
\item[{Args}] \leavevmode
\end{description}\end{quote}
\begin{itemize}
\item {} 
None

\end{itemize}
\begin{quote}\begin{description}
\item[{Returns}] \leavevmode
\end{description}\end{quote}
\begin{itemize}
\item {} 
A printed table including the node ID and it's coordinates

\end{itemize}

\end{fulllineitems}


\end{fulllineitems}



\section{MATERIAL}
\label{structures:material}\index{Material (class in AeroComBAT.Structures)}

\begin{fulllineitems}
\phantomsection\label{structures:AeroComBAT.Structures.Material}\pysiglinewithargsret{\strong{class }\code{AeroComBAT.Structures.}\bfcode{Material}}{\emph{MID}, \emph{name}, \emph{matType}, \emph{mat\_constants}, \emph{mat\_t}, \emph{**kwargs}}{}
creates a linear elastic material object.

This class creates a material object which can be stored within a
material library object. The material can be in general orthotropic.
\begin{quote}\begin{description}
\item[{Attributes}] \leavevmode
\end{description}\end{quote}
\begin{itemize}
\item {} 
\emph{name (str)}: A name for the material.

\item {} 
\emph{MID (int)}: An integer identifier for the material.

\item {} \begin{description}
\item[{\emph{matType (str)}: A string expressing what type of material it is.}] \leavevmode
Currently, the supported materials are isotropic, transversely
isotropic, and orthotropic.

\end{description}

\item {} \begin{description}
\item[{\emph{summary (str)}: A string which is a tabulated respresentation and}] \leavevmode
summary of the important attributes of the object.

\end{description}

\item {} \begin{description}
\item[{\emph{t (float)}: A single float which represents the thickness of a ply if}] \leavevmode
the material is to be used in a composite.

\end{description}

\item {} \begin{description}
\item[{\emph{rho (float)}: A signle float which represents the density of the}] \leavevmode
material.

\end{description}

\item {} \begin{description}
\item[{\emph{Smat (6x6 numpy Array{[}float{]})}: A numpy array representing the}] \leavevmode
compliance matrix in the fiber coordinate system.*

\end{description}

\item {} \begin{description}
\item[{\emph{Cmat (6x6 numpy Array{[}float{]})}: A numpy array representing the}] \leavevmode
stiffness matrix in the fiber coordinate system.*

\end{description}

\end{itemize}
\begin{quote}\begin{description}
\item[{Methods}] \leavevmode
\end{description}\end{quote}
\begin{itemize}
\item {} \begin{description}
\item[{\emph{printSummary}: This method prints out basic information about the}] \leavevmode
material, including the type, the material constants, material
thickness, as well as the tabulated stiffness or compliance
matricies if requested.

\end{description}

\end{itemize}

\begin{notice}{note}{Note:}
The CQUAD4 element assumes that the fibers are oriented along
the (1,0,0) in the global coordinate system.
\end{notice}
\index{\_\_init\_\_() (AeroComBAT.Structures.Material method)}

\begin{fulllineitems}
\phantomsection\label{structures:AeroComBAT.Structures.Material.__init__}\pysiglinewithargsret{\bfcode{\_\_init\_\_}}{\emph{MID}, \emph{name}, \emph{matType}, \emph{mat\_constants}, \emph{mat\_t}, \emph{**kwargs}}{}
Creates a material object

The main purpose of this class is assembling the constitutive
relations. Regardless of the analysis
\begin{quote}\begin{description}
\item[{Args}] \leavevmode
\end{description}\end{quote}
\begin{itemize}
\item {} 
\emph{MID (int)}: Material ID.

\item {} 
\emph{name (str)}: Name of the material.

\item {} \begin{description}
\item[{\emph{matType (str)}: The type of the material. Supported material types}] \leavevmode
are ``iso'', ``trans\_iso'', and ``ortho''.

\end{description}

\item {} \begin{description}
\item[{\emph{mat\_constants (1xX Array{[}Float{]})}: The requisite number of material}] \leavevmode
constants required for any structural analysis. Note, this
array includes the material density. For example, an isotropic
material needs 2 elastic material constants, so the total
length of mat\_constants would be 3, 2 elastic constants and the
density.

\end{description}

\item {} 
\emph{mat\_t (float)}: The thickness of 1-ply of the material

\item {} \begin{description}
\item[{\emph{th (1x3 Array{[}float{]})}: The angles about which the material can be}] \leavevmode
rotated when it is initialized. In degrees.

\end{description}

\end{itemize}
\begin{quote}\begin{description}
\item[{Returns}] \leavevmode
\end{description}\end{quote}
\begin{itemize}
\item {} 
None

\end{itemize}

\begin{notice}{note}{Note:}
While this class supports material direction rotations, it is more
robust to simply let the CQUAD4 and Mesher class handle all material
rotations.
\end{notice}

\end{fulllineitems}

\index{printSummary() (AeroComBAT.Structures.Material method)}

\begin{fulllineitems}
\phantomsection\label{structures:AeroComBAT.Structures.Material.printSummary}\pysiglinewithargsret{\bfcode{printSummary}}{\emph{**kwargs}}{}
Prints a tabulated summary of the material.

This method prints out basic information about the
material, including the type, the material constants, material
thickness, as well as the tabulated stiffness or compliance
matricies if requested.
\begin{quote}\begin{description}
\item[{Args}] \leavevmode
\end{description}\end{quote}
\begin{itemize}
\item {} \begin{description}
\item[{\emph{compliance (str)}: A boolean input to signify if the compliance}] \leavevmode
matrix should be printed.

\end{description}

\item {} \begin{description}
\item[{\emph{stiffness (str)}: A boolean input to signify if the stiffness matrix}] \leavevmode
should be printed.

\end{description}

\end{itemize}
\begin{quote}\begin{description}
\item[{Returns}] \leavevmode
\end{description}\end{quote}
\begin{itemize}
\item {} \begin{description}
\item[{String print out containing the material name, as well as material}] \leavevmode
constants and other defining material attributes. If requested
this includes the material stiffness and compliance matricies.

\end{description}

\end{itemize}

\end{fulllineitems}


\end{fulllineitems}



\section{CQUAD4}
\label{structures:cquad4}\index{CQUAD4 (class in AeroComBAT.Structures)}

\begin{fulllineitems}
\phantomsection\label{structures:AeroComBAT.Structures.CQUAD4}\pysiglinewithargsret{\strong{class }\code{AeroComBAT.Structures.}\bfcode{CQUAD4}}{\emph{EID}, \emph{nodes}, \emph{MID}, \emph{matLib}, \emph{**kwargs}}{}
Creates a linear, 2D 4 node quadrilateral element object.

The main purpose of this class is to assist in the cross-sectional
analysis of a beam, however it COULD be modified to serve as an element for
2D plate or laminate FE analysis.
\begin{quote}\begin{description}
\item[{Attributes}] \leavevmode
\end{description}\end{quote}
\begin{itemize}
\item {} 
\emph{type (str)}: A string designating it a CQUAD4 element.

\item {} \begin{description}
\item[{\emph{xsect (bool)}: States whether the element is to be used in cross-}] \leavevmode
sectional analysis.

\end{description}

\item {} \begin{description}
\item[{\emph{th (1x3 Array{[}float{]})}: Array containing the Euler-angles expressing how}] \leavevmode
the element constitutive relations should be rotated from the
material fiber frame to the global CSYS. In degrees.

\end{description}

\item {} 
\emph{EID (int)}: An integer identifier for the CQUAD4 element.

\item {} \begin{description}
\item[{\emph{MID (int)}: An integer refrencing the material ID used for the}] \leavevmode
constitutive relations.

\end{description}

\item {} \begin{description}
\item[{\emph{NIDs (1x4 Array{[}int{]})}: Contains the integer node identifiers for the}] \leavevmode
node objects used to create the element.

\end{description}

\item {} \begin{description}
\item[{\emph{nodes (1x4 Array{[}obj{]})}: Contains the properly ordered nodes objects}] \leavevmode
used to create the element.

\end{description}

\item {} \begin{description}
\item[{\emph{xs (1x4 np.array{[}float{]})}: Array containing the x-coordinates of the}] \leavevmode
nodes used in the element

\end{description}

\item {} \begin{description}
\item[{\emph{ys (1x4 np.array{[}float{]})}: Array containing the y-coordinates of the}] \leavevmode
nodes used in the element

\end{description}

\item {} 
\emph{rho (float)}: Density of the material used in the element.

\item {} 
\emph{mass (float)}: Mass per unit length (or thickness) of the element.

\item {} \begin{description}
\item[{\emph{U (12x1 np.array{[}float{]})}: This column vector contains the CQUAD4s}] \leavevmode
3 DOF (x-y-z) displacements in the local xsect CSYS due to cross-
section warping effects.

\end{description}

\item {} \begin{description}
\item[{\emph{Eps (6x4 np.array{[}float{]})}: A matrix containing the 3D strain state}] \leavevmode
within the CQUAD4 element.

\end{description}

\item {} \begin{description}
\item[{\emph{Sig (6x4 np.array{[}float{]})}: A matrix containing the 3D stress state}] \leavevmode
within the CQUAD4 element.

\end{description}

\end{itemize}
\begin{quote}\begin{description}
\item[{Methods}] \leavevmode
\end{description}\end{quote}
\begin{itemize}
\item {} \begin{description}
\item[{\emph{x}: Calculates the local xsect x-coordinate provided the desired master}] \leavevmode
coordinates eta and xi.

\end{description}

\item {} \begin{description}
\item[{\emph{y}: Calculates the local xsect y-coordinate provided the desired master}] \leavevmode
coordinates eta and xi.

\end{description}

\item {} \begin{description}
\item[{\emph{J}: Calculates the jacobian of the element provided the desired master}] \leavevmode
coordinates eta and xi.

\end{description}

\item {} \begin{description}
\item[{\emph{resetResults}: Initializes the displacement (U), strain (Eps), and}] \leavevmode
stress (Sig) attributes of the element.

\end{description}

\item {} \begin{description}
\item[{\emph{getDeformed}: Provided an analysis has been conducted, this method}] \leavevmode
returns 3 2x2 np.array{[}float{]} containing the element warped
displacements in the local xsect CSYS.

\end{description}

\item {} \begin{description}
\item[{\emph{getStressState}: Provided an analysis has been conducted, this method}] \leavevmode
returns 3 2x2 np.array{[}float{]} containing the element stress at four
points. The 3D stress state is processed to return the Von-Mises
or Maximum Principal stress state.

\end{description}

\item {} \begin{description}
\item[{\emph{printSummary}: Prints out a tabulated form of the element ID, as well}] \leavevmode
as the node ID's referenced by the element.

\end{description}

\end{itemize}
\index{J() (AeroComBAT.Structures.CQUAD4 method)}

\begin{fulllineitems}
\phantomsection\label{structures:AeroComBAT.Structures.CQUAD4.J}\pysiglinewithargsret{\bfcode{J}}{\emph{eta}, \emph{xi}}{}
Calculates the jacobian at a point in the element.

This method calculates the jacobian at a local point within the element
provided the master coordinates eta and xi.
\begin{quote}\begin{description}
\item[{Args}] \leavevmode
\end{description}\end{quote}
\begin{itemize}
\item {} 
\emph{eta (float)}: The eta coordinate in the master coordinate domain.*

\item {} 
\emph{xi (float)}: The xi coordinate in the master coordinate domain.*

\end{itemize}
\begin{quote}\begin{description}
\item[{Returns}] \leavevmode
\end{description}\end{quote}
\begin{itemize}
\item {} \begin{description}
\item[{\emph{Jmat (3x3 np.array{[}float{]})}: The stress-resutlant transformation}] \leavevmode
array.

\end{description}

\end{itemize}

\begin{notice}{note}{Note:}
Xi and eta can both vary between -1 and 1 respectively.
\end{notice}

\end{fulllineitems}

\index{\_\_init\_\_() (AeroComBAT.Structures.CQUAD4 method)}

\begin{fulllineitems}
\phantomsection\label{structures:AeroComBAT.Structures.CQUAD4.__init__}\pysiglinewithargsret{\bfcode{\_\_init\_\_}}{\emph{EID}, \emph{nodes}, \emph{MID}, \emph{matLib}, \emph{**kwargs}}{}
Initializes the element.
\begin{quote}\begin{description}
\item[{Args}] \leavevmode
\end{description}\end{quote}
\begin{itemize}
\item {} 
\emph{EID (int)}: An integer identifier for the CQUAD4 element.

\item {} \begin{description}
\item[{\emph{nodes (1x4 Array{[}obj{]})}: Contains the properly ordered nodes objects}] \leavevmode
used to create the element.

\end{description}

\item {} \begin{description}
\item[{\emph{MID (int)}: An integer refrencing the material ID used for the}] \leavevmode
constitutive relations.

\end{description}

\item {} \begin{description}
\item[{\emph{matLib (obj)}: A material library object containing a dictionary}] \leavevmode
with the material corresponding to the provided MID.

\end{description}

\item {} \begin{description}
\item[{\emph{xsect (bool)}: A boolean to determine whether this quad element is}] \leavevmode
to be usedfor cross-sectional analysis. Defualt value is True.

\end{description}

\item {} \begin{description}
\item[{\emph{th (1x3 Array{[}float{]})}: Array containing the Euler-angles expressing}] \leavevmode
how the element constitutive relations should be rotated from
the material fiber frame to the global CSYS. In degrees.

\end{description}

\end{itemize}
\begin{quote}\begin{description}
\item[{Returns}] \leavevmode
\end{description}\end{quote}
\begin{itemize}
\item {} 
None

\end{itemize}

\begin{notice}{note}{Note:}
The reference coordinate system for cross-sectional analysis is a
\end{notice}

local coordinate system in which the x and y axes are planer with the
element, and the z-axis is perpendicular to the plane of the element.

\end{fulllineitems}

\index{getDeformed() (AeroComBAT.Structures.CQUAD4 method)}

\begin{fulllineitems}
\phantomsection\label{structures:AeroComBAT.Structures.CQUAD4.getDeformed}\pysiglinewithargsret{\bfcode{getDeformed}}{\emph{**kwargs}}{}
Returns the warping displacement of the element.

Provided an analysis has been conducted, this method
returns 3 2x2 np.array{[}float{]} containing the element warped
displacements in the local xsect CSYS.
\begin{quote}\begin{description}
\item[{Args}] \leavevmode
\end{description}\end{quote}
\begin{itemize}
\item {} \begin{description}
\item[{\emph{warpScale (float)}: A multiplicative scaling factor intended to}] \leavevmode
exagerate the warping displacement within the cross-section.

\end{description}

\end{itemize}
\begin{quote}\begin{description}
\item[{Returns}] \leavevmode
\end{description}\end{quote}
\begin{itemize}
\item {} \begin{description}
\item[{\emph{xdef (2x2 np.array{[}float{]})}: warped x-coordinates at the four corner}] \leavevmode
points.

\end{description}

\item {} \begin{description}
\item[{\emph{ydef (2x2 np.array{[}float{]})}: warped y-coordinates at the four corner}] \leavevmode
points.

\end{description}

\item {} \begin{description}
\item[{\emph{zdef (2x2 np.array{[}float{]})}: warped z-coordinates at the four corner}] \leavevmode
points.

\end{description}

\end{itemize}

\end{fulllineitems}

\index{getStressState() (AeroComBAT.Structures.CQUAD4 method)}

\begin{fulllineitems}
\phantomsection\label{structures:AeroComBAT.Structures.CQUAD4.getStressState}\pysiglinewithargsret{\bfcode{getStressState}}{\emph{crit='VonMis'}}{}
Returns the stress state of the element.

Provided an analysis has been conducted, this method
returns a 2x2 np.array{[}float{]} containing the element the 3D stress
state at the four guass points by default.*
\begin{quote}\begin{description}
\item[{Args}] \leavevmode
\end{description}\end{quote}
\begin{itemize}
\item {} \begin{description}
\item[{\emph{crit (str)}: Determines what criteria is used to evaluate the 3D}] \leavevmode
stress state at the sample points within the element. By
default the Von Mises stress is returned. Currently supported
options include: Von Mises (`VonMis'), maximum principle stress
(`MaxPrin'), and the minimum principle stress (`MinPrin').

\end{description}

\end{itemize}
\begin{quote}\begin{description}
\item[{Returns}] \leavevmode
\end{description}\end{quote}
\begin{itemize}
\item {} \begin{description}
\item[{\emph{sigData (2x2 np.array{[}float{]})}: The stress state evaluated at four}] \leavevmode
points within the CQUAD4 element.

\end{description}

\end{itemize}

\begin{notice}{note}{Note:}
The XSect method calcWarpEffects is what determines where strain
\end{notice}

and stresses are sampled. By default it samples this information at the
Guass points where the stress/strain will be most accurate.

\end{fulllineitems}

\index{printSummary() (AeroComBAT.Structures.CQUAD4 method)}

\begin{fulllineitems}
\phantomsection\label{structures:AeroComBAT.Structures.CQUAD4.printSummary}\pysiglinewithargsret{\bfcode{printSummary}}{}{}
A method for printing a summary of the CQUAD4 element.

Prints out a tabulated form of the element ID, as well as the node ID's
referenced by the element.
\begin{quote}\begin{description}
\item[{Args}] \leavevmode
\end{description}\end{quote}
\begin{itemize}
\item {} 
None

\end{itemize}
\begin{quote}\begin{description}
\item[{Returns}] \leavevmode
\end{description}\end{quote}
\begin{itemize}
\item {} \begin{description}
\item[{(str): Prints the tabulated EID, node IDs and material IDs associated}] \leavevmode
with the CQUAD4 element.

\end{description}

\end{itemize}

\end{fulllineitems}

\index{resetResults() (AeroComBAT.Structures.CQUAD4 method)}

\begin{fulllineitems}
\phantomsection\label{structures:AeroComBAT.Structures.CQUAD4.resetResults}\pysiglinewithargsret{\bfcode{resetResults}}{}{}
Resets stress, strain and warping displacement results.

Method is mainly intended to prevent results for one analysis or
sampling location in the matrix to effect the results in another.
\begin{quote}\begin{description}
\item[{Args}] \leavevmode
\end{description}\end{quote}
\begin{itemize}
\item {} 
None

\end{itemize}
\begin{quote}\begin{description}
\item[{Returns}] \leavevmode
\end{description}\end{quote}
\begin{itemize}
\item {} 
None

\end{itemize}

\end{fulllineitems}

\index{x() (AeroComBAT.Structures.CQUAD4 method)}

\begin{fulllineitems}
\phantomsection\label{structures:AeroComBAT.Structures.CQUAD4.x}\pysiglinewithargsret{\bfcode{x}}{\emph{eta}, \emph{xi}}{}
Calculate the x-coordinate within the element.

Calculates the local xsect x-coordinate provided the desired master
coordinates eta and xi.
\begin{quote}\begin{description}
\item[{Args}] \leavevmode
\end{description}\end{quote}
\begin{itemize}
\item {} 
\emph{eta (float)}: The eta coordinate in the master coordinate domain.*

\item {} 
\emph{xi (float)}: The xi coordinate in the master coordinate domain.*

\end{itemize}
\begin{quote}\begin{description}
\item[{Returns}] \leavevmode
\end{description}\end{quote}
\begin{itemize}
\item {} 
\emph{float}: The x-coordinate within the element.

\end{itemize}

\begin{notice}{note}{Note:}
Xi and eta can both vary between -1 and 1 respectively.
\end{notice}

\end{fulllineitems}

\index{y() (AeroComBAT.Structures.CQUAD4 method)}

\begin{fulllineitems}
\phantomsection\label{structures:AeroComBAT.Structures.CQUAD4.y}\pysiglinewithargsret{\bfcode{y}}{\emph{eta}, \emph{xi}}{}
Calculate the y-coordinate within the element.

Calculates the local xsect y-coordinate provided the desired master
coordinates eta and xi.
\begin{quote}\begin{description}
\item[{Args}] \leavevmode
\end{description}\end{quote}
\begin{itemize}
\item {} 
\emph{eta (float)}: The eta coordinate in the master coordinate domain.*

\item {} 
\emph{xi (float)}: The xi coordinate in the master coordinate domain.*

\end{itemize}
\begin{quote}\begin{description}
\item[{Returns}] \leavevmode
\end{description}\end{quote}
\begin{itemize}
\item {} 
{\color{red}\bfseries{}{}`}(float)': The y-coordinate within the element.

\end{itemize}

\begin{notice}{note}{Note:}
Xi and eta can both vary between -1 and 1 respectively.
\end{notice}

\end{fulllineitems}


\end{fulllineitems}



\section{MATERIAL LIBRARY}
\label{structures:material-library}\index{MaterialLib (class in AeroComBAT.Structures)}

\begin{fulllineitems}
\phantomsection\label{structures:AeroComBAT.Structures.MaterialLib}\pysigline{\strong{class }\code{AeroComBAT.Structures.}\bfcode{MaterialLib}}
Creates a material library object.

This material library holds the materials to be used for any type of
analysis. Furthermore, it can be used to generate new material objects
to be automatically stored within it. See the Material class for suported
material types.
\begin{quote}\begin{description}
\item[{Attributes}] \leavevmode
\end{description}\end{quote}
\begin{itemize}
\item {} \begin{description}
\item[{\emph{matDict (dict)}: A dictionary which stores material objects as the}] \leavevmode
values with the MIDs as the associated keys.

\end{description}

\end{itemize}
\begin{quote}\begin{description}
\item[{Methods}] \leavevmode
\end{description}\end{quote}
\begin{itemize}
\item {} 
\emph{addMat}: Adds a material to the MaterialLib object dictionary.

\item {} 
\emph{getMat}: Returns a material object provided an MID

\item {} \begin{description}
\item[{\emph{printSummary}: Prints a summary of all of the materials held within the}] \leavevmode
matDict dictionary.

\end{description}

\end{itemize}
\index{\_\_init\_\_() (AeroComBAT.Structures.MaterialLib method)}

\begin{fulllineitems}
\phantomsection\label{structures:AeroComBAT.Structures.MaterialLib.__init__}\pysiglinewithargsret{\bfcode{\_\_init\_\_}}{}{}
Initialize MaterialLib object.

The initialization method is mainly used to initialize a dictionary
which houses material objects.
\begin{quote}\begin{description}
\item[{Args}] \leavevmode
\end{description}\end{quote}
\begin{itemize}
\item {} 
None

\end{itemize}
\begin{quote}\begin{description}
\item[{Returns}] \leavevmode
\end{description}\end{quote}
\begin{itemize}
\item {} 
None

\end{itemize}

\end{fulllineitems}

\index{addMat() (AeroComBAT.Structures.MaterialLib method)}

\begin{fulllineitems}
\phantomsection\label{structures:AeroComBAT.Structures.MaterialLib.addMat}\pysiglinewithargsret{\bfcode{addMat}}{\emph{MID}, \emph{mat\_name}, \emph{mat\_type}, \emph{mat\_constants}, \emph{mat\_t}, \emph{**kwargs}}{}
Add a material to the MaterialLib object.

This is the primary method of the class, used to create new material
obects and then add them to the library for later use.
\begin{quote}\begin{description}
\item[{Args}] \leavevmode
\end{description}\end{quote}
\begin{itemize}
\item {} 
\emph{MID (int)}: Material ID.

\item {} 
\emph{name (str)}: Name of the material.

\item {} \begin{description}
\item[{\emph{matType (str)}: The type of the material. Supported material types}] \leavevmode
are ``iso'', ``trans\_iso'', and ``ortho''.

\end{description}

\item {} \begin{description}
\item[{\emph{mat\_constants (1xX Array{[}Float{]})}: The requisite number of material}] \leavevmode
constants required for any structural analysis. Note, this
array includes the material density. For example, an isotropic
material needs 2 elastic material constants, so the total
length of mat\_constants would be 3, 2 elastic constants and the
density.

\end{description}

\item {} 
\emph{mat\_t (float)}: The thickness of 1-ply of the material

\item {} \begin{description}
\item[{\emph{th (1x3 Array{[}float{]})}: The angles about which the material can be}] \leavevmode
rotated when it is initialized. In degrees.

\end{description}

\item {} \begin{description}
\item[{\emph{overwrite (bool)}: Input used in order to define whether the}] \leavevmode
material being added can overwrite another material already
held by the material library with the same MID.

\end{description}

\end{itemize}
\begin{quote}\begin{description}
\item[{Returns}] \leavevmode
\end{description}\end{quote}
\begin{itemize}
\item {} 
None

\end{itemize}

\end{fulllineitems}

\index{getMat() (AeroComBAT.Structures.MaterialLib method)}

\begin{fulllineitems}
\phantomsection\label{structures:AeroComBAT.Structures.MaterialLib.getMat}\pysiglinewithargsret{\bfcode{getMat}}{\emph{MID}}{}
Method that returns a material from the material libary
\begin{quote}\begin{description}
\item[{Args}] \leavevmode
\end{description}\end{quote}
\begin{itemize}
\item {} 
\emph{MID (int)}: The ID of the material which is desired

\end{itemize}
\begin{quote}\begin{description}
\item[{Returns{}`}] \leavevmode
\end{description}\end{quote}
\begin{itemize}
\item {} 
{\color{red}\bfseries{}{}`}(obj): A material object associated with the key MID

\end{itemize}

\end{fulllineitems}

\index{printSummary() (AeroComBAT.Structures.MaterialLib method)}

\begin{fulllineitems}
\phantomsection\label{structures:AeroComBAT.Structures.MaterialLib.printSummary}\pysiglinewithargsret{\bfcode{printSummary}}{}{}
Prints summary of all Materials in MaterialLib

A method used to print out tabulated summary of all of the materials
held within the material library object.

Args:
- None

Returns:
- (str): A tabulated summary of the materials.

\end{fulllineitems}


\end{fulllineitems}



\section{PLY}
\label{structures:ply}\index{Ply (class in AeroComBAT.Structures)}

\begin{fulllineitems}
\phantomsection\label{structures:AeroComBAT.Structures.Ply}\pysiglinewithargsret{\strong{class }\code{AeroComBAT.Structures.}\bfcode{Ply}}{\emph{Material}, \emph{th}}{}
Creates a CLT ply object.

A class inspired by CLT, this class can be used to generate laminates
to be used for CLT or cross-sectional analysis. It is likely that ply
objects won't be created individually and then assembeled into a lamiante.
More likely is that the plies will be generated within the laminate object.
It should also be noted that it is assumed that the materials used are
effectively at most transversely isotropic.
\begin{quote}\begin{description}
\item[{Attributes}] \leavevmode
\end{description}\end{quote}
\begin{itemize}
\item {} 
\emph{E1 (float)}: Stiffness in the fiber direction.

\item {} 
\emph{E2 (float)}: Stiffness transverse to the fiber direction.

\item {} 
\emph{nu\_12 (float)}: In plane poisson ratio.

\item {} 
\emph{G\_12 (float)}: In plane shear modulus.

\item {} 
\emph{t (float)}: Thickness of the ply.

\item {} \begin{description}
\item[{\emph{Qbar (1x6 np.array{[}float{]})}: The terms in the rotated, reduced stiffness}] \leavevmode
matrix. Ordering is as follows: {[}Q11,Q12,Q16,Q22,Q26,Q66{]}

\end{description}

\item {} \begin{description}
\item[{\emph{MID (int)}: An integer refrencing the material ID used for the}] \leavevmode
constitutive relations.

\end{description}

\item {} \begin{description}
\item[{\emph{th (float)}: The angle about which the fibers are rotated in the plane}] \leavevmode
in degrees.

\end{description}

\end{itemize}
\begin{quote}\begin{description}
\item[{Methods}] \leavevmode
\end{description}\end{quote}
\begin{itemize}
\item {} \begin{description}
\item[{\emph{genQ}: Given the in-plane stiffnesses used by the material of the ply,}] \leavevmode
the method calculates the terms of ther reduced stiffness matrix.

\end{description}

\item {} \begin{description}
\item[{\emph{printSummary}: This prints out a summary of the object, including}] \leavevmode
thickness, referenced MID and in plane angle orientation theta in
degrees.

\end{description}

\end{itemize}
\index{\_\_init\_\_() (AeroComBAT.Structures.Ply method)}

\begin{fulllineitems}
\phantomsection\label{structures:AeroComBAT.Structures.Ply.__init__}\pysiglinewithargsret{\bfcode{\_\_init\_\_}}{\emph{Material}, \emph{th}}{}
Initializes the ply.

This method initializes information about the ply such as in-plane
stiffness repsonse.
\begin{quote}\begin{description}
\item[{Args}] \leavevmode
\end{description}\end{quote}
\begin{itemize}
\item {} \begin{description}
\item[{\emph{Material (obj)}: A material object, most likely coming from a}] \leavevmode
material library.

\end{description}

\item {} \begin{description}
\item[{\emph{th (float)}: The angle about which the fibers are rotated in the}] \leavevmode
plane in degrees.

\end{description}

\end{itemize}
\begin{quote}\begin{description}
\item[{Returns}] \leavevmode
\end{description}\end{quote}
\begin{itemize}
\item {} 
None

\end{itemize}

\end{fulllineitems}

\index{genQ() (AeroComBAT.Structures.Ply method)}

\begin{fulllineitems}
\phantomsection\label{structures:AeroComBAT.Structures.Ply.genQ}\pysiglinewithargsret{\bfcode{genQ}}{\emph{E1}, \emph{E2}, \emph{nu12}, \emph{G12}}{}
A method for calculating the reduced compliance of the ply.

Intended primarily as a private method but left public, this method,
for those unfarmiliar with CLT, calculates the terms in the reduced stiffness
matrix given the in plane ply stiffnesses. It can be thus inferred that
this requires the assumption of plane stres. This method is primarily
used during the ply instantiation.
\begin{quote}\begin{description}
\item[{Args}] \leavevmode
\end{description}\end{quote}
\begin{itemize}
\item {} 
\emph{E1 (float)}: The fiber direction stiffness.

\item {} 
\emph{E2 (float)}: The stiffness transverse to the fibers.

\item {} 
\emph{nu12 (float)}: The in-plane poisson ratio.

\item {} 
\emph{G12 (float)}: The in-plane shear stiffness.

\end{itemize}
\begin{quote}\begin{description}
\item[{Returns}] \leavevmode
\end{description}\end{quote}
\begin{itemize}
\item {} \begin{description}
\item[{\emph{(1x4 np.array{[}float{]})}: The terms used in the reduced stiffness}] \leavevmode
matrix. The ordering is: {[}Q11,Q12,Q22,Q66{]}.

\end{description}

\end{itemize}

\end{fulllineitems}

\index{printSummary() (AeroComBAT.Structures.Ply method)}

\begin{fulllineitems}
\phantomsection\label{structures:AeroComBAT.Structures.Ply.printSummary}\pysiglinewithargsret{\bfcode{printSummary}}{}{}
Prints a summary of the ply object.

A method for printing a summary of the ply properties, such as
the material ID, fiber orientation and ply thickness.
\begin{quote}\begin{description}
\item[{Args}] \leavevmode
\end{description}\end{quote}
\begin{itemize}
\item {} 
None

\end{itemize}
\begin{quote}\begin{description}
\item[{Returns}] \leavevmode
\end{description}\end{quote}
\begin{itemize}
\item {} 
{\color{red}\bfseries{}{}`}(str): Printed tabulated summary of the ply.

\end{itemize}

\end{fulllineitems}


\end{fulllineitems}



\section{LAMINATE}
\label{structures:laminate}\index{Laminate (class in AeroComBAT.Structures)}

\begin{fulllineitems}
\phantomsection\label{structures:AeroComBAT.Structures.Laminate}\pysiglinewithargsret{\strong{class }\code{AeroComBAT.Structures.}\bfcode{Laminate}}{\emph{n\_i\_tmp}, \emph{m\_i\_tmp}, \emph{matLib}, \emph{**kwargs}}{}
Creates a CLT laminate object.

This class has two main uses. It can either be used for CLT analysis, or it
can be used to build up a 2D mesh for a descretized cross-section.
\begin{quote}\begin{description}
\item[{Attributes}] \leavevmode
\end{description}\end{quote}
\begin{itemize}
\item {} \begin{description}
\item[{\emph{mesh (NxM np.array{[}int{]})}: This 2D array holds NIDs and is used}] \leavevmode
to represent how nodes are organized in the 2D cross-section of
the laminate.

\end{description}

\item {} \begin{description}
\item[{\emph{xmesh (NxM np.array{[}int{]})}: This 2D array holds the rigid x-coordinates}] \leavevmode
of the nodes within the 2D descretization of the laminate on the
local xsect CSYS.

\end{description}

\item {} \begin{description}
\item[{\emph{ymesh (NxM np.array{[}int{]})}: This 2D array holds the rigid y-coordinates}] \leavevmode
of the nodes within the 2D descretization of the laminate on the
local xsect CSYS.

\end{description}

\item {} \begin{description}
\item[{\emph{zmesh (NxM np.array{[}int{]})}: This 2D array holds the rigid z-coordinates}] \leavevmode
of the nodes within the 2D descretization of the laminate on the
local xsect CSYS.

\end{description}

\item {} 
\emph{H (float)}: The total laminate thickness.

\item {} 
\emph{rho\_A (float)}: The laminate area density.

\item {} \begin{description}
\item[{\emph{plies (1xN array{[}obj{]})}: Contains an array of ply objects used to}] \leavevmode
construct the laminate.

\end{description}

\item {} 
\emph{t (1xN array{[}float{]})}: An array containing all of the ply thicknesses.

\item {} \begin{description}
\item[{\emph{ABD (6x6 np.array{[}float{]})}: The CLT 6x6 matrix relating in-plane strains}] \leavevmode
and curvatures to in-plane force and moment resultants.

\end{description}

\item {} \begin{description}
\item[{\emph{abd (6x6 np.array{[}float{]})}: The CLT 6x6 matrix relating in-plane forces}] \leavevmode
and moments resultants to in-plane strains and curvatures.

\end{description}

\item {} \begin{description}
\item[{\emph{z (1xN array{[}float{]})}: The z locations of laminate starting and ending}] \leavevmode
points. This system always starts at -H/2 and goes to H/2

\end{description}

\item {} \begin{description}
\item[{\emph{equivMat (obj)}: This is orthotropic material object which exhibits}] \leavevmode
similar in-plane stiffnesses.

\end{description}

\item {} \begin{description}
\item[{\emph{forceRes (1x6 np.array{[}float{]})}: The applied or resulting force and}] \leavevmode
moment resultants generated during CLT analysis.

\end{description}

\item {} \begin{description}
\item[{\emph{globalStrain (1x6 np.array{[}float{]})}:  The applied or resulting strain}] \leavevmode
and curvatures generated during CLT analysis.

\end{description}

\end{itemize}
\begin{quote}\begin{description}
\item[{Methods}] \leavevmode
\end{description}\end{quote}
\begin{itemize}
\item {} \begin{description}
\item[{\emph{printSummary}: This method prints out defining attributes of the}] \leavevmode
laminate, such as the ABD matrix and layup schedule.

\end{description}

\end{itemize}
\index{\_\_init\_\_() (AeroComBAT.Structures.Laminate method)}

\begin{fulllineitems}
\phantomsection\label{structures:AeroComBAT.Structures.Laminate.__init__}\pysiglinewithargsret{\bfcode{\_\_init\_\_}}{\emph{n\_i\_tmp}, \emph{m\_i\_tmp}, \emph{matLib}, \emph{**kwargs}}{}
Initializes the Laminate object

The way the laminate initialization works is you pass in two-three
arrays and a material library. The first array contains information
about how many plies you want to stack, the second array determines
what material should be used for those plies, and the third array
determines at what angle those plies lie. The class was developed this
way as a means to fascilitate laminate optimization by quickly changing
the number of plies at a given orientation and using a given material.
\begin{quote}\begin{description}
\item[{Args}] \leavevmode
\end{description}\end{quote}
\begin{itemize}
\item {} \begin{description}
\item[{\emph{n\_i\_tmp (1xN array{[}int{]})}: An array containing the number of plies}] \leavevmode
using a material at a particular orientation such as:
(theta=0,theta=45...)

\end{description}

\item {} \begin{description}
\item[{\emph{m\_i\_tmp (1xN array{[}int{]})}: An array containing the material to be}] \leavevmode
used for the corresponding number of plies in the n\_i\_tmp array

\end{description}

\item {} \begin{description}
\item[{\emph{matLib (obj)}: The material library holding different material}] \leavevmode
objects.

\end{description}

\item {} 
\emph{sym (bool)}: Whether the laminate is symetric. (False by default)

\item {} \begin{description}
\item[{\emph{th (1xN array{[}float{]})}: An array containing the orientation at which}] \leavevmode
the fibers are positioned within the laminate.

\end{description}

\end{itemize}
\begin{quote}\begin{description}
\item[{Returns}] \leavevmode
\end{description}\end{quote}
\begin{itemize}
\item {} 
None

\end{itemize}

\begin{notice}{note}{Note:}
If you wanted to create a {[}0\_2/45\_2/90\_2/-45\_2{]}\_s laminate of the
same material, you could call laminate as:

lam = Laminate({[}2,2,2,2{]},{[}1,1,1,1{]},matLib,sym=True)

Or:

lam = Laminate({[}2,2,2,2{]},{[}1,1,1,1{]},matLib,sym=True,th={[}0,45,90,-45{]})

Both of these statements are equivalent. If no theta array is
provided and n\_i\_tmp is not equal to 4, then Laminate will default
your fibers to all be running in the 0 degree orientation.
\end{notice}

\end{fulllineitems}

\index{printSummary() (AeroComBAT.Structures.Laminate method)}

\begin{fulllineitems}
\phantomsection\label{structures:AeroComBAT.Structures.Laminate.printSummary}\pysiglinewithargsret{\bfcode{printSummary}}{\emph{**kwargs}}{}
Prints a summary of information about the laminate.

This method can print both the ABD matrix and ply information schedule
of the laminate.
\begin{quote}\begin{description}
\item[{Args}] \leavevmode
\end{description}\end{quote}
\begin{itemize}
\item {} \begin{description}
\item[{\emph{ABD (bool)}: This optional argument asks whether the ABD matrix}] \leavevmode
should be printed.

\end{description}

\item {} \begin{description}
\item[{\emph{decimals (int)}: Should the ABD matrix be printed, python should}] \leavevmode
print up to this many digits after the decimal point.

\end{description}

\item {} \begin{description}
\item[{\emph{plies (bool)}: This optional argument asks whether the ply schedule}] \leavevmode
for the laminate should be printed.

\end{description}

\end{itemize}
\begin{quote}\begin{description}
\item[{Returns}] \leavevmode
\end{description}\end{quote}
\begin{itemize}
\item {} 
None

\end{itemize}

\end{fulllineitems}


\end{fulllineitems}



\section{MESHER}
\label{structures:mesher}\index{Mesher (class in AeroComBAT.Structures)}

\begin{fulllineitems}
\phantomsection\label{structures:AeroComBAT.Structures.Mesher}\pysigline{\strong{class }\code{AeroComBAT.Structures.}\bfcode{Mesher}}
Meshes cross-section objects

This class is used to descritize cross-sections provided laminate objects.
Currently only two cross-sectional shapes are supported. The first is a
box beam using an airfoil outer mold line, and the second is a hollow tube
using as many laminates as desired. One of the main results is the
population of the nodeDict and elemDict attributes for the cross-section.
\begin{quote}\begin{description}
\item[{Attributes}] \leavevmode
\end{description}\end{quote}
\begin{itemize}
\item {} 
None

\end{itemize}
\begin{quote}\begin{description}
\item[{Methods}] \leavevmode
\end{description}\end{quote}
\begin{itemize}
\item {} \begin{description}
\item[{\emph{boxBeam}: Taking several inputs including 4 laminate objects and meshes}] \leavevmode
a 2D box beam cross-section.

\end{description}

\item {} \begin{description}
\item[{\emph{cylindricalTube}: Taking several inputs including n laminate objects and}] \leavevmode
meshes a 2D cylindrical tube cross-section.

\end{description}

\end{itemize}
\index{boxBeam() (AeroComBAT.Structures.Mesher method)}

\begin{fulllineitems}
\phantomsection\label{structures:AeroComBAT.Structures.Mesher.boxBeam}\pysiglinewithargsret{\bfcode{boxBeam}}{\emph{xsect}, \emph{meshSize}, \emph{x0}, \emph{xf}, \emph{matlib}}{}
Meshes a box beam cross-section.

This method is currently the only supported and tested meshing method.
The mesher assumes that the laminates in the box beam are oriented such
that the top surfaces of the laminate are facing inwards. Therefore if
you would like a particular fiber orientation on the outter or inner
most surfaces, create your laminate layup schedules appropriately.
\begin{quote}\begin{description}
\item[{Args}] \leavevmode
\end{description}\end{quote}
\begin{itemize}
\item {} 
\emph{xsect (obj)}: The cross-section object to be meshed.

\item {} 
\emph{meshSize (int)}: The maximum aspect ratio an element can have

\item {} \begin{description}
\item[{\emph{x0 (float)}: The non-dimensional starting point of the cross-section}] \leavevmode
on the airfoil.

\end{description}

\item {} \begin{description}
\item[{\emph{xf (float)}: The non-dimesnional ending point of the cross-section}] \leavevmode
on the airfoil.

\end{description}

\item {} \begin{description}
\item[{\emph{matlib (obj)}: The material library object used to create CQUAD4}] \leavevmode
elements.

\end{description}

\end{itemize}
\begin{quote}\begin{description}
\item[{Returns}] \leavevmode
\end{description}\end{quote}
\begin{itemize}
\item {} 
None

\end{itemize}

\end{fulllineitems}


\end{fulllineitems}



\section{CROSS-SECTION}
\label{structures:cross-section}\index{XSect (class in AeroComBAT.Structures)}

\begin{fulllineitems}
\phantomsection\label{structures:AeroComBAT.Structures.XSect}\pysiglinewithargsret{\strong{class }\code{AeroComBAT.Structures.}\bfcode{XSect}}{\emph{XID}, \emph{Airfoil}, \emph{xdim}, \emph{laminates}, \emph{matlib}, \emph{**kwargs}}{}
Creates a beam cross-section object,

This cross-section can be made of multiple materials which can be in
general anisotropic. This is the main workhorse within the structures
library.
\begin{quote}\begin{description}
\item[{Attributes}] \leavevmode
\end{description}\end{quote}
\begin{itemize}
\item {} \begin{description}
\item[{\emph{Color (touple)}: A length 3 touple used to define the color of the}] \leavevmode
cross-section.

\end{description}

\item {} \begin{description}
\item[{\emph{Airfoil (obj)}: The airfoil object used to define the OML of the cross-}] \leavevmode
section.

\end{description}

\item {} \begin{description}
\item[{\emph{typeXSect (str)}: Defines what type of cross-section is to be used.}] \leavevmode
Currently the only supported type is `box'.

\end{description}

\item {} \begin{description}
\item[{\emph{normalVector (1x3 np.array{[}float{]})}: Expresses the normal vector of the}] \leavevmode
cross-section.

\end{description}

\item {} \begin{description}
\item[{\emph{nodeDict (dict)}: A dictionary of all nodes used to descretize the}] \leavevmode
cross-section surface. The keys are the NIDs and the values stored
are the Node objects.

\end{description}

\item {} \begin{description}
\item[{\emph{elemDict (dict)}: A dictionary of all elements used to descretize the}] \leavevmode
cross-section surface. the keys are the EIDs and the values stored
are the element objects.

\end{description}

\item {} \begin{description}
\item[{\emph{X (ndx6 np.array{[}float{]})}: A very large 2D array. This is one of the}] \leavevmode
results of the cross-sectional analysis. This array relays the
force and moment resultants applied to the cross-section to the
nodal warping displacements exhibited by the cross-section.

\end{description}

\item {} \begin{description}
\item[{\emph{Y (6x6 np.array{[}float{]})}: This array relays the force and moment}] \leavevmode
resultants applied to the cross-section to the rigid section
strains and curvatures exhibited by the cross-section.

\end{description}

\item {} \begin{description}
\item[{\emph{dXdz (ndx6 np.array{[}float{]})}: A very large 2D array. This is one of the}] \leavevmode
results of the cross-sectional analysis. This array relays the
force and moment resultants applied to the cross-section to the
gradient of the nodal warping displacements exhibited by the
cross-section with respect to the beam axis.

\end{description}

\item {} \begin{description}
\item[{\emph{xt (float)}: The x-coordinate of the tension center (point at which}] \leavevmode
tension and bending are decoupled)

\end{description}

\item {} \begin{description}
\item[{\emph{yt (float)}: The y-coordinate of the tension center (point at which}] \leavevmode
tension and bending are decoupled)

\end{description}

\item {} \begin{description}
\item[{\emph{xs (float)}: The x-coordinate of the shear center (point at which shear}] \leavevmode
and torsion are decoupled)

\end{description}

\item {} \begin{description}
\item[{\emph{ys (float)}: The y-coordinate of the shear center (point at which shear}] \leavevmode
and torsion are decoupled)

\end{description}

\item {} \begin{description}
\item[{\emph{refAxis (3x1 np.array{[}float{]})}: A column vector containing the reference}] \leavevmode
axis for the beam.

\end{description}

\item {} \begin{description}
\item[{\emph{bendAxes (2x3 np.array{[}float{]})}: Contains two row vectors about which}] \leavevmode
bending from one axis is decoupled from bending about the other.

\end{description}

\item {} \begin{description}
\item[{\emph{F\_raw (6x6 np.array{[}float{]})}: The 6x6 compliance matrix that results}] \leavevmode
from cross-sectional analysis. This is the case where the reference
axis is at the origin.

\end{description}

\item {} \begin{description}
\item[{\emph{K\_raw (6x6 np.array{[}float{]})}: The 6x6 stiffness matrix that results}] \leavevmode
from cross-sectional analysis. This is the case where the reference
axis is at the origin.

\end{description}

\item {} \begin{description}
\item[{\emph{F (6x6 np.array{[}float{]})}: The 6x6 compliance matrix for the cross-}] \leavevmode
section about the reference axis. The reference axis is by default
at the shear center.

\end{description}

\item {} \begin{description}
\item[{\emph{K (6x6 np.array{[}float{]})}: The 6x6 stiffness matrix for the cross-}] \leavevmode
section about the reference axis. The reference axis is by default
at the shear center.

\end{description}

\item {} \begin{description}
\item[{\emph{T1 (3x6 np.array{[}float{]})}: The transformation matrix that converts}] \leavevmode
strains and curvatures from the local xsect origin to the reference
axis.

\end{description}

\item {} \begin{description}
\item[{\emph{T2 (3x6 np.array{[}float{]})}: The transformation matrix that converts}] \leavevmode
forces and moments from the local xsect origin to the reference
axis.

\end{description}

\item {} \begin{description}
\item[{\emph{x\_m (1x3 np.array{[}float{]})}: Center of mass of the cross-section about in}] \leavevmode
the local xsect CSYS

\end{description}

\item {} \begin{description}
\item[{\emph{M (6x6 np.array{[}float{]})}: This mass matrix relays linear and angular}] \leavevmode
velocities to linear and angular momentum of the cross-section.

\end{description}

\end{itemize}
\begin{quote}\begin{description}
\item[{Methods}] \leavevmode
\end{description}\end{quote}
\begin{itemize}
\item {} \begin{description}
\item[{\emph{resetResults}: This method resets all results (displacements, strains}] \leavevmode
and stresse) within the elements used by the cross-section object.

\end{description}

\item {} \begin{description}
\item[{\emph{calcWarpEffects}: Given applied force and moment resultants, this method}] \leavevmode
calculates the warping displacement, 3D strains and 3D stresses
within the elements used by the cross-section.

\end{description}

\item {} \begin{description}
\item[{\emph{printSummary}: This method is used to print characteristic attributes of}] \leavevmode
the object. This includes the elastic, shear and mass centers, as
well as the stiffness matrix and mass matrix.

\end{description}

\item {} \begin{description}
\item[{\emph{plotRigid}: This method plots the rigid cross-section shape, typically}] \leavevmode
in conjunction with a full beam model.

\end{description}

\item {} \begin{description}
\item[{\emph{plotWarped}: This method plots the warped cross-section including a}] \leavevmode
contour criteria, typically in conjuction with the results of the
displacement of a full beam model.

\end{description}

\end{itemize}
\index{\_\_init\_\_() (AeroComBAT.Structures.XSect method)}

\begin{fulllineitems}
\phantomsection\label{structures:AeroComBAT.Structures.XSect.__init__}\pysiglinewithargsret{\bfcode{\_\_init\_\_}}{\emph{XID}, \emph{Airfoil}, \emph{xdim}, \emph{laminates}, \emph{matlib}, \emph{**kwargs}}{}
Instantiates a cross-section object.

The constructor for the class is effectively responsible for creating
the 2D desretized mesh of the cross-section. It is important to note
that while meshing technically occurs in the constructor, the work is
handeled by another class altogether. While not
computationally heavily intensive in itself, it is responsible for
creating all of the framework for the cross-sectional analysis.
\begin{quote}\begin{description}
\item[{Args}] \leavevmode
\end{description}\end{quote}
\begin{itemize}
\item {} 
\emph{XID (int)}: The cross-section integer identifier.

\item {} \begin{description}
\item[{\emph{Airfoil (obj)}: An airfoil object used to determine the OML shape of}] \leavevmode
the cross-section.

\end{description}

\item {} \begin{description}
\item[{\emph{xdim (1x2 array{[}float{]})}: The non-dimensional starting and stoping}] \leavevmode
points of the cross-section. In other words, if you wanted to
have your cross-section start at the 1/4 chord and run to the
3/4 chord of your airfoil, xdim would look like xdim={[}0.25,0.75{]}

\end{description}

\item {} \begin{description}
\item[{\emph{laminates (1xN array{[}obj{]})}: Laminate objects used to create the}] \leavevmode
descretized mesh surface. Do not repeat a laminate within this
array! It will referrence this object multiple times and not
mesh the cross-section properly then!

\end{description}

\item {} 
\emph{matlib (obj)}: A material library

\item {} \begin{description}
\item[{\emph{typeXSect (str)}: The general shape the cross-section should take.}] \leavevmode
Note that currently only a box beam profile is supported.
More shapes and the ability to add stiffeners to the
cross-section will come in later updates.

\end{description}

\item {} \begin{description}
\item[{\emph{meshSize (int)}: The maximum aspect ratio you would like your 2D}] \leavevmode
CQUAD4 elements to exhibit within the cross-section.

\end{description}

\end{itemize}
\begin{quote}\begin{description}
\item[{Returns}] \leavevmode
\end{description}\end{quote}
\begin{itemize}
\item {} 
None

\end{itemize}

\end{fulllineitems}

\index{calcWarpEffects() (AeroComBAT.Structures.XSect method)}

\begin{fulllineitems}
\phantomsection\label{structures:AeroComBAT.Structures.XSect.calcWarpEffects}\pysiglinewithargsret{\bfcode{calcWarpEffects}}{\emph{**kwargs}}{}
Calculates displacements, stresses, and strains for applied forces

The second most powerful method of the XSect class. After an analysis
is run, the FEM class stores force and moment resultants within the
beam element objects. From there, warping displacement, strain and
stress can be determined within the cross-section at any given location
within the beam using this method. This method will take a while though
as it has to calculate 4 displacements and 24 stresses and strains for
every element within the cross-section. Keep that in mind when you are
surveying your beam or wing for displacements, stresses and strains.
\begin{quote}\begin{description}
\item[{Args}] \leavevmode
\end{description}\end{quote}
\begin{itemize}
\item {} \begin{description}
\item[{\emph{force (6x1 np.array{[}float{]})}: This is the internal force and moment}] \leavevmode
resultant experienced by the cross-section.

\end{description}

\end{itemize}
\begin{quote}\begin{description}
\item[{Returns}] \leavevmode
\end{description}\end{quote}
\begin{itemize}
\item {} 
None

\end{itemize}

\end{fulllineitems}

\index{plotRigid() (AeroComBAT.Structures.XSect method)}

\begin{fulllineitems}
\phantomsection\label{structures:AeroComBAT.Structures.XSect.plotRigid}\pysiglinewithargsret{\bfcode{plotRigid}}{\emph{**kwargs}}{}
Plots the rigid cross-section along a beam.

This method is very useful for visually debugging a structural model.
It will plot out the rigid cross-section in 3D space with regards to
the reference axis.
\begin{quote}\begin{description}
\item[{Args}] \leavevmode
\end{description}\end{quote}
\begin{itemize}
\item {} \begin{description}
\item[{\emph{x (1x3 np.array{[}float{]})}: The rigid location on your beam you are}] \leavevmode
trying to plot:

\end{description}

\item {} \begin{description}
\item[{\emph{beam\_axis (1x3 np.array{[}float{]})}: The vector pointing in the}] \leavevmode
direction of your beam axis.

\end{description}

\item {} 
\emph{figName (str)}: The name of the figure.

\item {} \begin{description}
\item[{\emph{wireMesh (bool)}: A boolean to determine of the wiremesh outline}] \leavevmode
should be plotted.*

\end{description}

\end{itemize}
\begin{quote}\begin{description}
\item[{Returns}] \leavevmode
\end{description}\end{quote}
\begin{itemize}
\item {} 
\emph{(fig)}: Plots the cross-section in a mayavi figure.

\end{itemize}

\begin{notice}{note}{Note:}
Because of how the mayavi wireframe keyword works, it will
\end{notice}

apear as though the cross-section is made of triangles as opposed to
quadrilateras. Fear not! They are made of quads, the wireframe is just
plotted as triangles.

\end{fulllineitems}

\index{plotWarped() (AeroComBAT.Structures.XSect method)}

\begin{fulllineitems}
\phantomsection\label{structures:AeroComBAT.Structures.XSect.plotWarped}\pysiglinewithargsret{\bfcode{plotWarped}}{\emph{**kwargs}}{}
Plots the warped cross-section along a beam.

Once an analysis has been completed, this method can be utilized in
order to plot the results anywhere along the beam.
\begin{quote}\begin{description}
\item[{Args}] \leavevmode
\end{description}\end{quote}
\begin{itemize}
\item {} \begin{description}
\item[{\emph{displScale (float)}: The scale by which all rotations and}] \leavevmode
displacements will be mutliplied in order make it visually
easier to detect displacements.

\end{description}

\item {} \begin{description}
\item[{\emph{x (1x3 np.array{[}float{]})}: The rigid location on your beam you are}] \leavevmode
trying to plot:

\end{description}

\item {} \begin{description}
\item[{\emph{U (1x6 np.array{[}float{]})}: The rigid body displacements and rotations}] \leavevmode
experienced by the cross-section.

\end{description}

\item {} \begin{description}
\item[{\emph{beam\_axis (1x3 np.array{[}float{]})}: The vector pointing in the}] \leavevmode
direction of your beam axis.

\end{description}

\item {} \begin{description}
\item[{\emph{contour (str)}: Determines what value is to be plotted during as a}] \leavevmode
contour in the cross-section.

\end{description}

\item {} 
\emph{figName (str)}: The name of the figure.

\item {} \begin{description}
\item[{\emph{wireMesh (bool)}: A boolean to determine of the wiremesh outline}] \leavevmode
should be plotted.*

\end{description}

\item {} \begin{description}
\item[{\emph{contLim (1x2 array{[}float{]})}: Describes the upper and lower bounds of}] \leavevmode
contour color scale.

\end{description}

\item {} \begin{description}
\item[{\emph{warpScale (float)}: The scaling factor by which all warping}] \leavevmode
displacements in the cross-section will be multiplied.

\end{description}

\end{itemize}
\begin{quote}\begin{description}
\item[{Returns}] \leavevmode
\end{description}\end{quote}
\begin{itemize}
\item {} 
\emph{(fig)}: Plots the cross-section in a mayavi figure.

\end{itemize}

\begin{notice}{note}{Note:}
Because of how the mayavi wireframe keyword works, it will
\end{notice}

apear as though the cross-section is made of triangles as opposed to
quadrilateras. Fear not! They are made of quads, the wireframe is just
plotted as triangles.

\end{fulllineitems}

\index{printSummary() (AeroComBAT.Structures.XSect method)}

\begin{fulllineitems}
\phantomsection\label{structures:AeroComBAT.Structures.XSect.printSummary}\pysiglinewithargsret{\bfcode{printSummary}}{\emph{refAxis=True}, \emph{decimals=8}, \emph{**kwargs}}{}
Print characterisic information about the cross-section.

This method prints out characteristic information about the cross-
section objects. By default, the method will print out the location of
the reference axis, the shear, tension, and mass center. This method
if requested will also print the stiffness and mass matricies.
\begin{quote}\begin{description}
\item[{Args}] \leavevmode
\end{description}\end{quote}
\begin{itemize}
\item {} \begin{description}
\item[{\emph{refAxis (bool)}: Boolean to determine if the stiffness matrix}] \leavevmode
printed should be about the reference axis (True) or about the
local xsect origin (False).

\end{description}

\item {} \begin{description}
\item[{\emph{stiffMat (bool)}: Boolean to determine if the stiffness matrix}] \leavevmode
should be printed.

\end{description}

\item {} \begin{description}
\item[{\emph{tensCntr (bool)}: Boolean to determine if the location of the tension}] \leavevmode
center should be printed.

\end{description}

\item {} \begin{description}
\item[{\emph{shearCntr (bool)}: Boolean to determine if the location of the shear}] \leavevmode
center should be printed.

\end{description}

\item {} \begin{description}
\item[{\emph{massCntr (bool)}: Boolean to determine if the location of the mass}] \leavevmode
center should be printed.

\end{description}

\item {} \begin{description}
\item[{\emph{refAxisLoc (bool)}: Boolean to determine if the location of the}] \leavevmode
reference axis should be printed.

\end{description}

\end{itemize}
\begin{quote}\begin{description}
\item[{Returns}] \leavevmode
\end{description}\end{quote}
\begin{itemize}
\item {} 
\emph{(str)}: Prints out a string of information about the cross-section.

\end{itemize}

\end{fulllineitems}


\end{fulllineitems}



\section{TIMOSHENKO BEAM}
\label{structures:timoshenko-beam}\index{TBeam (class in AeroComBAT.Structures)}

\begin{fulllineitems}
\phantomsection\label{structures:AeroComBAT.Structures.TBeam}\pysiglinewithargsret{\strong{class }\code{AeroComBAT.Structures.}\bfcode{TBeam}}{\emph{x1}, \emph{x2}, \emph{xsect}, \emph{EID=0}, \emph{SBID=0}, \emph{nid1=0}, \emph{nid2=1}}{}
Creates a Timoshenko beam finite element object.

The primary beam finite element used by AeroComBAT, this beam element is
similar to the Euler-Bernoulli beam finite element most are farmiliar with,
with the exception that it has the ability to experience shear deformation
in addition to just bending.
\begin{quote}\begin{description}
\item[{Attributes}] \leavevmode
\end{description}\end{quote}
\begin{itemize}
\item {} 
\emph{type (str)}:String describing the type of beam element being used.

\item {} \begin{description}
\item[{\emph{U1 (dict)}: This dictionary contains the results of an analysis set. The}] \leavevmode
keys are the string names of the analysis and the values stored are
6x1 np.array{[}float{]} vectors containing the 3 displacements and
3 rotations at the first node.

\end{description}

\item {} \begin{description}
\item[{\emph{U2 (dict)}: This dictionary contains the results of an analysis set. The}] \leavevmode
keys are the string names of the analysis and the values stored are
6x1 np.array{[}float{]} vectors containing the 3 displacements and
3 rotations at the second node.

\end{description}

\item {} \begin{description}
\item[{\emph{Umode1 (dict)}: This dictionary contains the results of a modal analysis}] \leavevmode
set. The keys are the string names of the analysis and the values
stored are 6xN np.array{[}float{]}. The columns of the array are the
displacements and rotations at the first node associated with the
particular mode.

\end{description}

\item {} \begin{description}
\item[{\emph{Umode2 (dict)}: This dictionary contains the results of a modal analysis}] \leavevmode
set. The keys are the string names of the analysis and the values
stored are 6xN np.array{[}float{]}. The columns of the array are the
displacements and rotations at the second node associated with the
particular mode.

\end{description}

\item {} \begin{description}
\item[{\emph{F1 (dict)}: This dictionary contains the results of an analysis set. The}] \leavevmode
keys are the string names of the analysis and the values stored are
6x1 np.array{[}float{]} vectors containing the 3 internal forces and
3 moments at the first node.

\end{description}

\item {} \begin{description}
\item[{\emph{F2 (dict)}: This dictionary contains the results of an analysis set. The}] \leavevmode
keys are the string names of the analysis and the values stored are
6x1 np.array{[}float{]} vectors containing the 3 internal forces and
3 moments at the second node.

\end{description}

\item {} \begin{description}
\item[{\emph{Fmode1 (dict)}: This dictionary contains the results of a modal analysis}] \leavevmode
set. The keys are the string names of the analysis and the values
stored are 6xN np.array{[}float{]}. The columns of the array are the
forces and moments at the first node associated with the
particular mode.*

\end{description}

\item {} \begin{description}
\item[{\emph{Fmode2 (dict)}: This dictionary contains the results of a modal analysis}] \leavevmode
set. The keys are the string names of the analysis and the values
stored are 6xN np.array{[}float{]}. The columns of the array are the
forces and moments at the second node associated with the
particular mode.*

\end{description}

\item {} \begin{description}
\item[{\emph{xsect (obj)}: The cross-section object used to determine the beams}] \leavevmode
stiffnesses.

\end{description}

\item {} 
\emph{EID (int)}: The element ID of the beam.

\item {} 
\emph{SBID (int)}: The associated Superbeam ID the beam object belongs to.

\item {} 
\emph{n1 (obj)}: The first nodal object used by the beam.

\item {} 
\emph{n2 (obj)}: The second nodal object used by the beam.

\item {} 
\emph{Fe (12x1 np.array{[}float{]})}: The distributed force vector of the element

\item {} 
\emph{Ke (12x12 np.array{[}float{]})}: The stiffness matrix of the beam.

\item {} \begin{description}
\item[{\emph{Keg (12x12 np.array{[}float{]})}: The geometric stiffness matrix of the}] \leavevmode
beam. Used for beam buckling calculations.

\end{description}

\item {} 
\emph{Me (12x12 np.array{[}float{]})}: The mass matrix of the beam.

\item {} 
\emph{h (float)}: The magnitude length of the beam element.

\item {} \begin{description}
\item[{\emph{xbar (float)}: The unit vector pointing in the direction of the rigid}] \leavevmode
beam.

\end{description}

\end{itemize}
\begin{quote}\begin{description}
\item[{Methods}] \leavevmode
\end{description}\end{quote}
\begin{itemize}
\item {} \begin{description}
\item[{\emph{printSummary}: This method prints out characteristic attributes of the}] \leavevmode
beam finite element.

\end{description}

\item {} 
\emph{plotRigidBeam}: Plots the the shape of the rigid beam element.

\item {} 
\emph{plotDisplBeam}: Plots the deformed shape of the beam element.

\item {} \begin{description}
\item[{\emph{printInternalForce}: Prints the internal forces of the beam element for}] \leavevmode
a given analysis set

\end{description}

\end{itemize}

\begin{notice}{note}{Note:}
The force and moments in the Fmode1 and Fmode2 could be completely
\end{notice}

fictitious and be left as an artifact to fascilitate plotting of warped
cross-sections. DO NOT rely on this information being meaningful.
\index{\_\_init\_\_() (AeroComBAT.Structures.TBeam method)}

\begin{fulllineitems}
\phantomsection\label{structures:AeroComBAT.Structures.TBeam.__init__}\pysiglinewithargsret{\bfcode{\_\_init\_\_}}{\emph{x1}, \emph{x2}, \emph{xsect}, \emph{EID=0}, \emph{SBID=0}, \emph{nid1=0}, \emph{nid2=1}}{}
Instantiates a timoshenko beam element.

This method instatiates a finite element timoshenko beam element.
Currently the beam must be oriented along the global y-axis, however
full 3D orientation support for frames is in progress.
\begin{quote}\begin{description}
\item[{Args}] \leavevmode
\end{description}\end{quote}
\begin{itemize}
\item {} \begin{description}
\item[{\emph{x1 (1x3 np.array{[}float{]})}: The 3D coordinates of the first beam}] \leavevmode
element node.

\end{description}

\item {} \begin{description}
\item[{\emph{x2 (1x3 np.array{[}float{]})}: The 3D coordinates of the second beam}] \leavevmode
element node.

\end{description}

\item {} \begin{description}
\item[{\emph{xsect (obj)}: The cross-section object used to determine stiffnes}] \leavevmode
and mass properties for the beam.

\end{description}

\item {} 
\emph{EID (int)}: The integer identifier for the beam.

\item {} 
\emph{SBID (int)}: The associated superbeam ID.

\item {} 
\emph{nid1 (int)}: The first node ID

\item {} 
\emph{nid2 (int)}: The second node ID

\end{itemize}
\begin{quote}\begin{description}
\item[{Returns}] \leavevmode
\end{description}\end{quote}
\begin{itemize}
\item {} 
None

\end{itemize}

\end{fulllineitems}

\index{plotDisplBeam() (AeroComBAT.Structures.TBeam method)}

\begin{fulllineitems}
\phantomsection\label{structures:AeroComBAT.Structures.TBeam.plotDisplBeam}\pysiglinewithargsret{\bfcode{plotDisplBeam}}{\emph{**kwargs}}{}
Plots the displaced beam in 3D space.

This method plots the deformed beam finite element in 3D space. It is
not typically called by the beam object but by a SuperBeam object
or even a WingSection object.
\begin{quote}\begin{description}
\item[{Args}] \leavevmode
\end{description}\end{quote}
\begin{itemize}
\item {} \begin{description}
\item[{\emph{environment (str)}: Determines what environment is to be used to}] \leavevmode
plot the beam in 3D space. Currently only mayavi is supported.

\end{description}

\item {} 
\emph{figName (str)}: The name of the figure in which the beam will apear.

\item {} \begin{description}
\item[{\emph{clr (1x3 touple(float))}: This touple contains three floats running}] \leavevmode
from 0 to 1 in order to generate a color mayavi can plot.

\end{description}

\item {} \begin{description}
\item[{\emph{displScale (float)}: The scaling factor for the deformation}] \leavevmode
experienced by the beam.

\end{description}

\item {} \begin{description}
\item[{\emph{mode (int)}: Determines what mode to plot. By default the mode is 0}] \leavevmode
implying a non-eigenvalue solution should be plotted.

\end{description}

\end{itemize}
\begin{quote}\begin{description}
\item[{Returns}] \leavevmode
\end{description}\end{quote}
\begin{itemize}
\item {} 
\emph{(fig)}: The mayavi figure of the beam.

\end{itemize}

\end{fulllineitems}

\index{plotRigidBeam() (AeroComBAT.Structures.TBeam method)}

\begin{fulllineitems}
\phantomsection\label{structures:AeroComBAT.Structures.TBeam.plotRigidBeam}\pysiglinewithargsret{\bfcode{plotRigidBeam}}{\emph{**kwargs}}{}
Plots the rigid beam in 3D space.

This method plots the beam finite element in 3D space. It is not
typically called by the beam object but by a SuperBeam object or
even a WingSection object.
\begin{quote}\begin{description}
\item[{Args}] \leavevmode
\end{description}\end{quote}
\begin{itemize}
\item {} \begin{description}
\item[{\emph{environment (str)}: Determines what environment is to be used to}] \leavevmode
plot the beam in 3D space. Currently only mayavi is supported.

\end{description}

\item {} 
\emph{figName (str)}: The name of the figure in which the beam will apear.

\item {} \begin{description}
\item[{\emph{clr (1x3 touple(float))}: This touple contains three floats running}] \leavevmode
from 0 to 1 in order to generate a color mayavi can plot.

\end{description}

\end{itemize}
\begin{quote}\begin{description}
\item[{Returns}] \leavevmode
\end{description}\end{quote}
\begin{itemize}
\item {} 
\emph{(fig)}: The mayavi figure of the beam.

\end{itemize}

\end{fulllineitems}

\index{printInternalForce() (AeroComBAT.Structures.TBeam method)}

\begin{fulllineitems}
\phantomsection\label{structures:AeroComBAT.Structures.TBeam.printInternalForce}\pysiglinewithargsret{\bfcode{printInternalForce}}{\emph{**kwargs}}{}
Prints the internal forces and moments in the beam.

For a particular analysis set, this method prints out the force and
moment resultants at both nodes of the beam.
\begin{quote}\begin{description}
\item[{Args}] \leavevmode
\end{description}\end{quote}
\begin{itemize}
\item {} \begin{description}
\item[{\emph{analysis\_name (str)}: The analysis name for which the forces are}] \leavevmode
being surveyed.

\end{description}

\end{itemize}
\begin{quote}\begin{description}
\item[{Returns}] \leavevmode
\end{description}\end{quote}
\begin{itemize}
\item {} \begin{description}
\item[{\emph{(str)}: This is a print out of the internal forces and moments}] \leavevmode
within the beam element.

\end{description}

\end{itemize}

\end{fulllineitems}

\index{printSummary() (AeroComBAT.Structures.TBeam method)}

\begin{fulllineitems}
\phantomsection\label{structures:AeroComBAT.Structures.TBeam.printSummary}\pysiglinewithargsret{\bfcode{printSummary}}{\emph{decimals=8}, \emph{**kwargs}}{}
Prints out characteristic information about the beam element.

This method by default prints out the EID, XID, SBID and the NIDs along
with the nodes associated coordinates. Upon request, it can also print
out the beam element stiffness, geometric stiffness, mass matricies and
distributed force vector.
\begin{quote}\begin{description}
\item[{Args}] \leavevmode
\end{description}\end{quote}
\begin{itemize}
\item {} \begin{description}
\item[{\emph{nodeCoord (bool)}: A boolean to determine if the node coordinate}] \leavevmode
information should also be printed.

\end{description}

\item {} \begin{description}
\item[{\emph{Ke (bool)}: A boolean to determine if the element stiffness matrix}] \leavevmode
should be printed.

\end{description}

\item {} \begin{description}
\item[{\emph{Keg (bool)}: A boolean to determine if the element gemoetric}] \leavevmode
stiffness matrix should be printed.

\end{description}

\item {} \begin{description}
\item[{\emph{Me (bool)}: A boolean to determine if the element mass matrix}] \leavevmode
should be printed.

\end{description}

\item {} \begin{description}
\item[{\emph{Fe (bool)}: A boolean to determine if the element distributed force}] \leavevmode
and moment vector should be printed.

\end{description}

\end{itemize}
\begin{quote}\begin{description}
\item[{Returns}] \leavevmode
\end{description}\end{quote}
\begin{itemize}
\item {} 
\emph{(str)}: Printed summary of the requested attributes.

\end{itemize}

\end{fulllineitems}


\end{fulllineitems}



\section{SUPER-BEAM}
\label{structures:super-beam}\index{SuperBeam (class in AeroComBAT.Structures)}

\begin{fulllineitems}
\phantomsection\label{structures:AeroComBAT.Structures.SuperBeam}\pysiglinewithargsret{\strong{class }\code{AeroComBAT.Structures.}\bfcode{SuperBeam}}{\emph{x1}, \emph{x2}, \emph{xsect}, \emph{noe}, \emph{SBID}, \emph{btype='Tbeam'}, \emph{sNID=1}, \emph{sEID=1}}{}
Create a superbeam object.

The superbeam object is mainly to fascilitate creating a whole series of
beam objects along  the same line.
\begin{quote}\begin{description}
\item[{Attributes}] \leavevmode
\end{description}\end{quote}
\begin{itemize}
\item {} 
\emph{type (str)}: The object type, a `SuperBeam'.

\item {} 
\emph{btype (str)}: The beam element type of the elements in the superbeam.

\item {} 
\emph{SBID (int)}: The integer identifier for the superbeam.

\item {} 
\emph{sNID (int)}: The starting NID of the superbeam.

\item {} 
\emph{enid (int)}: The ending NID of the superbeam.

\item {} \begin{description}
\item[{\emph{xsect (obj)}: The cross-section object referenced by the beam elements}] \leavevmode
in the superbeam.

\end{description}

\item {} 
\emph{noe (int)}: Number of elements in the beam.

\item {} 
\emph{NIDs2EIDs (dict)}: Mapping of NIDs to beam EIDs within the superbeam

\item {} \begin{description}
\item[{\emph{x1 (1x3 np.array{[}float{]})}: The 3D coordinate of the first point on the}] \leavevmode
superbeam.

\end{description}

\item {} \begin{description}
\item[{\emph{x2 (1x3 np.array{[}float{]})}: The 3D coordinate of the last point on the}] \leavevmode
superbeam.

\end{description}

\item {} \begin{description}
\item[{\emph{sEID (int)}: The integer identifier for the first beam element in the}] \leavevmode
superbeam.

\end{description}

\item {} \begin{description}
\item[{\emph{elems (dict)}: A dictionary of all beam elements within the superbeam.}] \leavevmode
The keys are the EIDs and the values are the corresponding beam
elements.

\end{description}

\item {} \begin{description}
\item[{\emph{xbar (1x3 np.array{[}float{]})}: The vector pointing along the axis of the}] \leavevmode
superbeam.

\end{description}

\end{itemize}
\begin{quote}\begin{description}
\item[{Methods}] \leavevmode
\end{description}\end{quote}
\begin{itemize}
\item {} 
\emph{getBeamCoord}: Returns the 3D coordinate of a point along the superbeam.

\item {} \begin{description}
\item[{\emph{printInternalForce}: Prints all internal forces and moments at every}] \leavevmode
node in the superbeam.

\end{description}

\item {} \begin{description}
\item[{\emph{writeDisplacements}: Writes all displacements and rotations in the}] \leavevmode
superbeam to a .csv

\end{description}

\item {} \begin{description}
\item[{\emph{getEIDatx}: Provided a non-dimensional point along the superbeam, this}] \leavevmode
method returns the local element EID and the non-dimensional
coordinate within that element.

\end{description}

\item {} \begin{description}
\item[{\emph{printSummary}: Prints all of the elements and node IDs within the beam}] \leavevmode
as well as the coordinates of those nodes.

\end{description}

\end{itemize}
\index{\_\_init\_\_() (AeroComBAT.Structures.SuperBeam method)}

\begin{fulllineitems}
\phantomsection\label{structures:AeroComBAT.Structures.SuperBeam.__init__}\pysiglinewithargsret{\bfcode{\_\_init\_\_}}{\emph{x1}, \emph{x2}, \emph{xsect}, \emph{noe}, \emph{SBID}, \emph{btype='Tbeam'}, \emph{sNID=1}, \emph{sEID=1}}{}
Creates a superelement object.

This method instantiates a superelement. What it effectively does is
mesh a line provided the starting and ending points along that line.
Keep in mind that for now, only beams running parallel to the z-axis
are supported.
\begin{quote}\begin{description}
\item[{Args}] \leavevmode
\end{description}\end{quote}
\begin{itemize}
\item {} 
\emph{x1 (1x3 np.array{[}float{]})}: The starting coordinate of the beam.

\item {} 
\emph{x2 (1x3 np.array{[}float{]})}: The ending coordinate of the beam.

\item {} 
\emph{xsect (obj)}: The cross-section used throught the superbeam.

\item {} 
\emph{noe (int)}: The number of elements along the beam.

\item {} 
\emph{SBID (int)}: The integer identifier for the superbeam.

\item {} \begin{description}
\item[{\emph{btype (str)}: The beam type to be meshed. Currently only Tbeam types}] \leavevmode
are supported.

\end{description}

\item {} 
\emph{sNID (int)}: The starting NID for the superbeam.

\item {} 
\emph{sEID (int)}: The starting EID for the superbeam.

\end{itemize}
\begin{quote}\begin{description}
\item[{Returns}] \leavevmode
\end{description}\end{quote}
\begin{itemize}
\item {} 
None

\end{itemize}

\end{fulllineitems}

\index{getBeamCoord() (AeroComBAT.Structures.SuperBeam method)}

\begin{fulllineitems}
\phantomsection\label{structures:AeroComBAT.Structures.SuperBeam.getBeamCoord}\pysiglinewithargsret{\bfcode{getBeamCoord}}{\emph{x\_nd}}{}
Determine the global coordinate along superbeam.

Provided the non-dimensional coordinate along the beam, this method
returns the global coordinate at that point.
\begin{quote}\begin{description}
\item[{Args}] \leavevmode
\end{description}\end{quote}
\begin{itemize}
\item {} \begin{description}
\item[{\emph{x\_nd (float)}: The non-dimensional coordinate along the beam. Note}] \leavevmode
that x\_nd must be between zero and one.

\end{description}

\end{itemize}
\begin{quote}\begin{description}
\item[{Returns}] \leavevmode
\end{description}\end{quote}
\begin{itemize}
\item {} 
\emph{(1x3 np.array{[}float{]})}: The global coordinate corresponding to x\_nd

\end{itemize}

\end{fulllineitems}

\index{getEIDatx() (AeroComBAT.Structures.SuperBeam method)}

\begin{fulllineitems}
\phantomsection\label{structures:AeroComBAT.Structures.SuperBeam.getEIDatx}\pysiglinewithargsret{\bfcode{getEIDatx}}{\emph{x}}{}
Returns the beam EID at a non-dimensional x-location in the superbeam.

Provided the non-dimensional coordinate along the beam, this method
returns the global beam element EID, as well as the local non-
dimensional coordinate within the specific beam element.
\begin{quote}\begin{description}
\item[{Args}] \leavevmode
\end{description}\end{quote}
\begin{itemize}
\item {} 
\emph{x (float)}: The non-dimensional coordinate within the super-beam

\end{itemize}
\begin{quote}\begin{description}
\item[{Returns}] \leavevmode
\end{description}\end{quote}
\begin{itemize}
\item {} \begin{description}
\item[{\emph{EID (int)}: The EID of the element containing the non-dimensional}] \leavevmode
coordinate provided.

\end{description}

\item {} \begin{description}
\item[{\emph{local\_x\_nd (float)}: The non-dimensional coordinate within the beam}] \leavevmode
element associated with the provided non-dimensional coordinate
within the beam.

\end{description}

\end{itemize}

\end{fulllineitems}

\index{printInternalForce() (AeroComBAT.Structures.SuperBeam method)}

\begin{fulllineitems}
\phantomsection\label{structures:AeroComBAT.Structures.SuperBeam.printInternalForce}\pysiglinewithargsret{\bfcode{printInternalForce}}{\emph{**kwargs}}{}
Prints the internal forces and moments in the superbeam.

For every node within the superbeam, this method will print out the
internal forces and moments at those nodes.
\begin{quote}\begin{description}
\item[{Args}] \leavevmode
\end{description}\end{quote}
\begin{itemize}
\item {} \begin{description}
\item[{\emph{analysis\_name (str)}: The name of the analysis for which the forces}] \leavevmode
and moments are being surveyed.

\end{description}

\end{itemize}
\begin{quote}\begin{description}
\item[{Returns}] \leavevmode
\end{description}\end{quote}
\begin{itemize}
\item {} 
\emph{(str)}: Printed output expressing all forces and moments.

\end{itemize}

\end{fulllineitems}

\index{printSummary() (AeroComBAT.Structures.SuperBeam method)}

\begin{fulllineitems}
\phantomsection\label{structures:AeroComBAT.Structures.SuperBeam.printSummary}\pysiglinewithargsret{\bfcode{printSummary}}{\emph{decimals=8}, \emph{**kwargs}}{}
Prints out characteristic information about the super beam.

This method by default prints out the EID, XID, SBID and the NIDs along
with the nodes associated coordinates. Upon request, it can also print
out the beam element stiffness, geometric stiffness, mass matricies and
distributed force vector.
\begin{quote}\begin{description}
\item[{Args}] \leavevmode
\end{description}\end{quote}
\begin{itemize}
\item {} \begin{description}
\item[{\emph{nodeCoord (bool)}: A boolean to determine if the node coordinate}] \leavevmode
information should also be printed.

\end{description}

\item {} \begin{description}
\item[{\emph{Ke (bool)}: A boolean to determine if the element stiffness matrix}] \leavevmode
should be printed.

\end{description}

\item {} \begin{description}
\item[{\emph{Keg (bool)}: A boolean to determine if the element gemoetric}] \leavevmode
stiffness matrix should be printed.

\end{description}

\item {} \begin{description}
\item[{\emph{Me (bool)}: A boolean to determine if the element mass matrix}] \leavevmode
should be printed.

\end{description}

\item {} \begin{description}
\item[{\emph{Fe (bool)}: A boolean to determine if the element distributed force}] \leavevmode
and moment vector should be printed.

\end{description}

\end{itemize}
\begin{quote}\begin{description}
\item[{Returns}] \leavevmode
\end{description}\end{quote}
\begin{itemize}
\item {} 
\emph{(str)}: Printed summary of the requested attributes.

\end{itemize}

\end{fulllineitems}

\index{writeDisplacements() (AeroComBAT.Structures.SuperBeam method)}

\begin{fulllineitems}
\phantomsection\label{structures:AeroComBAT.Structures.SuperBeam.writeDisplacements}\pysiglinewithargsret{\bfcode{writeDisplacements}}{\emph{**kwargs}}{}
Write internal displacements and rotations to file.

For every node within the superbeam, this method will tabulate all of
the displacements and rotations and then write them to a file.
\begin{quote}\begin{description}
\item[{Args}] \leavevmode
\end{description}\end{quote}
\begin{itemize}
\item {} 
\emph{fileName (str)}: The name of the file where the data will be written.

\item {} \begin{description}
\item[{\emph{analysis\_name (str)}: The name of the analysis for which the}] \leavevmode
displacements and rotations are being surveyed.

\end{description}

\end{itemize}
\begin{quote}\begin{description}
\item[{Returns}] \leavevmode
\end{description}\end{quote}
\begin{itemize}
\item {} \begin{description}
\item[{\emph{fileName (file)}: This method doesn't actually return a file, rather}] \leavevmode
it writes the data to a file named ``fileName'' and saves it to the
working directory.

\end{description}

\end{itemize}

\end{fulllineitems}

\index{writeForcesMoments() (AeroComBAT.Structures.SuperBeam method)}

\begin{fulllineitems}
\phantomsection\label{structures:AeroComBAT.Structures.SuperBeam.writeForcesMoments}\pysiglinewithargsret{\bfcode{writeForcesMoments}}{\emph{**kwargs}}{}
Write internal force and moments to file.

For every node within the superbeam, this method will tabulate all of
the forces and moments and then write them to a file.
\begin{quote}\begin{description}
\item[{Args}] \leavevmode
\end{description}\end{quote}
\begin{itemize}
\item {} 
\emph{fileName (str)}: The name of the file where the data will be written.

\item {} \begin{description}
\item[{\emph{analysis\_name (str)}: The name of the analysis for which the}] \leavevmode
forces and moments are being surveyed.

\end{description}

\end{itemize}
\begin{quote}\begin{description}
\item[{Returns}] \leavevmode
\end{description}\end{quote}
\begin{itemize}
\item {} \begin{description}
\item[{\emph{fileName (file)}: This method doesn't actually return a file, rather}] \leavevmode
it writes the data to a file named ``fileName'' and saves it to the
working directory.

\end{description}

\end{itemize}

\end{fulllineitems}


\end{fulllineitems}



\section{WING SECTION}
\label{structures:wing-section}\index{WingSection (class in AeroComBAT.Structures)}

\begin{fulllineitems}
\phantomsection\label{structures:AeroComBAT.Structures.WingSection}\pysiglinewithargsret{\strong{class }\code{AeroComBAT.Structures.}\bfcode{WingSection}}{\emph{x1}, \emph{x2}, \emph{chord}, \emph{name}, \emph{x0\_spar}, \emph{xf\_spar}, \emph{laminates}, \emph{matLib}, \emph{noe}, \emph{SSBID=0}, \emph{SNID=0}, \emph{SEID=0}, \emph{**kwargs}}{}
Creates a wing section object.

This class instantiates a wing section object which is intended to
represent the section of a wing enclosed by two ribs. This allows primarily
for two different things: it allows the user to vary the cross-section
design of the wing by enabling different designs in each wing section, as
well as enabling the user to estimate the static stability of the laminates
that make up the wing-section design.
\begin{quote}\begin{description}
\item[{Attributes}] \leavevmode
\end{description}\end{quote}
\begin{itemize}
\item {} \begin{description}
\item[{\emph{Airfoils (Array{[}obj{]})}: This array contains all of the airfoils used}] \leavevmode
over the wing section. This attribute exists primarily to fascilitate
the meshing process and is subject to change.

\end{description}

\item {} \begin{description}
\item[{\emph{XSects (Array{[}obj{]})}: This array contains all of the cross-section}] \leavevmode
objects used in the wing section. If the cross-section is constant
along the length of the wing section, this array length is 1.

\end{description}

\item {} \begin{description}
\item[{\emph{SuperBeams (Array{[}obj{]})}: This array contains all of the superbeam}] \leavevmode
objects used in the wing section. If the cross-section is constant
along the length of the wing section, this array length is 1.

\end{description}

\item {} \begin{description}
\item[{\emph{xdim (1x2 Array{[}float{]})}: This array contains the non-dimensional}] \leavevmode
starting and ending points of the wing section spar. They are
non-dimensionalized by the chord length.

\end{description}

\item {} \begin{description}
\item[{\emph{Laminates (Array{[}obj{]})}: This array contains the laminate objects used}] \leavevmode
by the cross-sections in the wing section.

\end{description}

\item {} 
\emph{x1 (1x3 np.array{[}float{]})}: The starting coordinate of the wing section.

\item {} 
\emph{x2 (1x3 np.array{[}float{]})}: The ending coordinate of the wing section.

\item {} 
\emph{XIDs (Array{[}int{]})}: This array containts the integer cross-section IDs

\end{itemize}
\begin{quote}\begin{description}
\item[{Methods}] \leavevmode
\end{description}\end{quote}
\begin{itemize}
\item {} 
\emph{plotRigid}: This method plots the rigid wing section in 3D space.

\item {} \begin{description}
\item[{\emph{plotDispl}: Provided an analysis name, this method will deformed state}] \leavevmode
of the wing section. It is also capable of plotting cross-section
criteria, such as displacement, stress, strain, or failure criteria.

\end{description}

\end{itemize}

\begin{notice}{warning}{Warning:}
While it is possible to use multiple cross-section within the
wing section, this capability is only to be utilized for tapering cross
sections, not changing the cross-section type or design (such as by
changing the laminates used to make the cross-sections). Doing so would
invalidate the ritz method buckling solutions applied to the laminate
objects.
\end{notice}
\index{\_\_init\_\_() (AeroComBAT.Structures.WingSection method)}

\begin{fulllineitems}
\phantomsection\label{structures:AeroComBAT.Structures.WingSection.__init__}\pysiglinewithargsret{\bfcode{\_\_init\_\_}}{\emph{x1}, \emph{x2}, \emph{chord}, \emph{name}, \emph{x0\_spar}, \emph{xf\_spar}, \emph{laminates}, \emph{matLib}, \emph{noe}, \emph{SSBID=0}, \emph{SNID=0}, \emph{SEID=0}, \emph{**kwargs}}{}
Creates a wing section object

This wing section object is in some way an organizational object. It
holds a collection of superbeam objects which in general could all use
different cross-sections. One could for example use several super-beams
in order to simlate a taper within a wing section descretely. These
objects will also be used in order to determine the buckling span of
the laminate objects held within the cross-section.
\begin{quote}\begin{description}
\item[{Args}] \leavevmode
\end{description}\end{quote}
\begin{itemize}
\item {} \begin{description}
\item[{\emph{x1 (1x3 np.array{[}float{]})}: The starting coordinate of the wing}] \leavevmode
section.

\end{description}

\item {} \begin{description}
\item[{\emph{x2 (1x3 np.array{[}float{]})}: The ending coordinate of the wing}] \leavevmode
section.

\end{description}

\item {} \begin{description}
\item[{\emph{chord (func)}: A function that returns the chord length along a wing}] \leavevmode
provided the scalar length from the wing origin to the desired
point.

\end{description}

\item {} \begin{description}
\item[{\emph{name (str)}: The name of the airfoil to be used to mesh the}] \leavevmode
cross-section. This is subject to change since the meshing process
is only a placeholder.

\end{description}

\item {} \begin{description}
\item[{\emph{x0\_spar (float)}: The non-dimensional starting location of the cross}] \leavevmode
section. This value is non-dimensionalized by the local chord
length.

\end{description}

\item {} \begin{description}
\item[{\emph{xf\_spar (float)}: The non-dimensional ending location of the cross}] \leavevmode
section. This value is non-dimensionalized by the local chord
length.

\end{description}

\item {} \begin{description}
\item[{\emph{laminates (Array{[}obj{]})}: This array contains the laminate objects to}] \leavevmode
be used in order to mesh the cross-section.

\end{description}

\item {} \begin{description}
\item[{\emph{matLib (obj)}: This material library object contains all of the}] \leavevmode
materials to be used in meshing the cross-sections used by the
wing section.

\end{description}

\item {} \begin{description}
\item[{\emph{noe (float)}: The number of beam elements to be used in the wing per}] \leavevmode
unit length.

\end{description}

\item {} 
\emph{SSBID (int)}: The starting superbeam ID in the wing section.

\item {} 
\emph{SNID (int)}: The starting node ID in the wing section.

\item {} 
\emph{SEID (int)}: The starting element ID in the wing section.

\item {} 
\emph{SXID (int)}: The starting cross-section ID in the wing section.

\item {} \begin{description}
\item[{\emph{numSupBeams (int)}: The number of different superbeams to be used}] \leavevmode
in the wing section.

\end{description}

\item {} \begin{description}
\item[{\emph{typeXSect (str)}: The type of cross-section used by the wing}] \leavevmode
section.

\end{description}

\item {} \begin{description}
\item[{\emph{meshSize (int)}: The maximum aspect ratio an element can have within}] \leavevmode
the cross-sections used by the wing sections.

\end{description}

\item {} \begin{description}
\item[{\emph{ref\_ax (str)}: The reference axis used by the cross-section. This is}] \leavevmode
axis about which the loads will be applied on the wing section.

\end{description}

\end{itemize}

\begin{notice}{note}{Note:}
The chord function could take the shape of: 
chord = lambda y: (ctip-croot)*y/b\_s+croot
\end{notice}

\end{fulllineitems}

\index{plotDispl() (AeroComBAT.Structures.WingSection method)}

\begin{fulllineitems}
\phantomsection\label{structures:AeroComBAT.Structures.WingSection.plotDispl}\pysiglinewithargsret{\bfcode{plotDispl}}{\emph{**kwargs}}{}
Plots the deformed wing section object in 3D space.

Provided an analysis name, this method will plot the results from the
corresponding analysis including beam/cross-section deformation, and
stress, strain, or failure criteria within the sampled cross-sections.
\begin{quote}\begin{description}
\item[{Args}] \leavevmode
\end{description}\end{quote}
\begin{itemize}
\item {} \begin{description}
\item[{\emph{figName (str)}: The name of the plot to be generated. If one is not}] \leavevmode
provided a semi-random name will be generated.

\end{description}

\item {} \begin{description}
\item[{\emph{environment (str)}: The name of the environment to be used when}] \leavevmode
plotting. Currently only the `mayavi' environment is supported.

\end{description}

\item {} \begin{description}
\item[{\emph{clr (1x3 tuple(int))}: This tuple represents the RGB values that the}] \leavevmode
beam reference axis will be colored with.

\end{description}

\item {} \begin{description}
\item[{\emph{numXSects (int)}: This is the number of cross-sections that will be}] \leavevmode
plotted and evenly distributed throughout the beam.

\end{description}

\item {} \begin{description}
\item[{\emph{contour (str)}: The contour to be plotted on the sampled cross}] \leavevmode
sections.

\end{description}

\item {} \begin{description}
\item[{\emph{contLim (1x2 Array{[}float{]})}: The lower and upper limits for the}] \leavevmode
contour color plot.

\end{description}

\item {} \begin{description}
\item[{\emph{warpScale (float)}: The visual multiplication factor to be applied}] \leavevmode
to the cross-sectional warping displacement.

\end{description}

\item {} \begin{description}
\item[{\emph{displScale (float)}: The visual multiplication factor to be applied}] \leavevmode
to the beam displacements and rotations.

\end{description}

\item {} \begin{description}
\item[{\emph{analysis\_name (str)}: The analysis name corresponding to the results}] \leavevmode
to pe visualized.

\end{description}

\item {} \begin{description}
\item[{\emph{mode (int)}: For modal analysis, this corresponds to the mode-shape}] \leavevmode
which is desired to be plotted.

\end{description}

\end{itemize}
\begin{quote}\begin{description}
\item[{Returns}] \leavevmode
\end{description}\end{quote}
\begin{itemize}
\item {} 
\emph{(figure)}: This method returns a 3D plot of the rigid wing section.

\end{itemize}

\begin{notice}{warning}{Warning:}
In order to limit the size of data stored in memory, the
local cross-sectional data is not stored. As a result, for every
additional cross-section that is plotted, the time required to plot
will increase substantially.
\end{notice}

\end{fulllineitems}

\index{plotRigid() (AeroComBAT.Structures.WingSection method)}

\begin{fulllineitems}
\phantomsection\label{structures:AeroComBAT.Structures.WingSection.plotRigid}\pysiglinewithargsret{\bfcode{plotRigid}}{\emph{**kwargs}}{}
Plots the rigid wing section object in 3D space.

This method is exceptionally helpful when building up a model and
debugging it.
\begin{quote}\begin{description}
\item[{Args}] \leavevmode
\end{description}\end{quote}
\begin{itemize}
\item {} \begin{description}
\item[{\emph{figName (str)}: The name of the plot to be generated. If one is not}] \leavevmode
provided a semi-random name will be generated.

\end{description}

\item {} \begin{description}
\item[{\emph{environment (str)}: The name of the environment to be used when}] \leavevmode
plotting. Currently only the `mayavi' environment is supported.

\end{description}

\item {} \begin{description}
\item[{\emph{clr (1x3 tuple(int))}: This tuple represents the RGB values that the}] \leavevmode
beam reference axis will be colored with.

\end{description}

\item {} \begin{description}
\item[{\emph{numXSects (int)}: This is the number of cross-sections that will be}] \leavevmode
plotted and evenly distributed throughout the beam.

\end{description}

\end{itemize}
\begin{quote}\begin{description}
\item[{Returns}] \leavevmode
\end{description}\end{quote}
\begin{itemize}
\item {} 
\emph{(figure)}: This method returns a 3D plot of the rigid wing section.

\end{itemize}

\begin{notice}{warning}{Warning:}
In order to limit the size of data stored in memory, the
local cross-sectional data is not stored. As a result, for every
additional cross-section that is plotted, the time required to plot
will increase substantially.
\end{notice}

\end{fulllineitems}


\end{fulllineitems}



\chapter{Indices and tables}
\label{index:indices-and-tables}\begin{itemize}
\item {} 
\DUspan{xref,std,std-ref}{genindex}

\item {} 
\DUspan{xref,std,std-ref}{modindex}

\item {} 
\DUspan{xref,std,std-ref}{search}

\end{itemize}


\renewcommand{\indexname}{Python Module Index}
\begin{theindex}
\def\bigletter#1{{\Large\sffamily#1}\nopagebreak\vspace{1mm}}
\bigletter{a}
\item {\texttt{AeroComBAT.FEM}}, \pageref{FEM:module-AeroComBAT.FEM}
\item {\texttt{AeroComBAT.Structures}}, \pageref{structures:module-AeroComBAT.Structures}
\end{theindex}

\renewcommand{\indexname}{Index}
\printindex
\end{document}
